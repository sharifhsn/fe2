\documentclass[11pt]{article}

    \usepackage[breakable]{tcolorbox}
    \usepackage{parskip} % Stop auto-indenting (to mimic markdown behaviour)
    

    % Basic figure setup, for now with no caption control since it's done
    % automatically by Pandoc (which extracts ![](path) syntax from Markdown).
    \usepackage{graphicx}
    % Keep aspect ratio if custom image width or height is specified
    \setkeys{Gin}{keepaspectratio}
    % Maintain compatibility with old templates. Remove in nbconvert 6.0
    \let\Oldincludegraphics\includegraphics
    % Ensure that by default, figures have no caption (until we provide a
    % proper Figure object with a Caption API and a way to capture that
    % in the conversion process - todo).
    \usepackage{caption}
    \DeclareCaptionFormat{nocaption}{}
    \captionsetup{format=nocaption,aboveskip=0pt,belowskip=0pt}

    \usepackage{float}
    \floatplacement{figure}{H} % forces figures to be placed at the correct location
    \usepackage{xcolor} % Allow colors to be defined
    \usepackage{enumerate} % Needed for markdown enumerations to work
    \usepackage{geometry} % Used to adjust the document margins
    \usepackage{amsmath} % Equations
    \usepackage{amssymb} % Equations
    \usepackage{textcomp} % defines textquotesingle
    % Hack from http://tex.stackexchange.com/a/47451/13684:
    \AtBeginDocument{%
        \def\PYZsq{\textquotesingle}% Upright quotes in Pygmentized code
    }
    \usepackage{upquote} % Upright quotes for verbatim code
    \usepackage{eurosym} % defines \euro

    \usepackage{iftex}
    \ifPDFTeX
        \usepackage[T1]{fontenc}
        \IfFileExists{alphabeta.sty}{
              \usepackage{alphabeta}
          }{
              \usepackage[mathletters]{ucs}
              \usepackage[utf8x]{inputenc}
          }
    \else
        \usepackage{fontspec}
        \usepackage{unicode-math}
    \fi

    \usepackage{fancyvrb} % verbatim replacement that allows latex
    \usepackage{grffile} % extends the file name processing of package graphics
                         % to support a larger range
    \makeatletter % fix for old versions of grffile with XeLaTeX
    \@ifpackagelater{grffile}{2019/11/01}
    {
      % Do nothing on new versions
    }
    {
      \def\Gread@@xetex#1{%
        \IfFileExists{"\Gin@base".bb}%
        {\Gread@eps{\Gin@base.bb}}%
        {\Gread@@xetex@aux#1}%
      }
    }
    \makeatother
    \usepackage[Export]{adjustbox} % Used to constrain images to a maximum size
    \adjustboxset{max size={0.9\linewidth}{0.9\paperheight}}

    % The hyperref package gives us a pdf with properly built
    % internal navigation ('pdf bookmarks' for the table of contents,
    % internal cross-reference links, web links for URLs, etc.)
    \usepackage{hyperref}
    % The default LaTeX title has an obnoxious amount of whitespace. By default,
    % titling removes some of it. It also provides customization options.
    \usepackage{titling}
    \usepackage{longtable} % longtable support required by pandoc >1.10
    \usepackage{booktabs}  % table support for pandoc > 1.12.2
    \usepackage{array}     % table support for pandoc >= 2.11.3
    \usepackage{calc}      % table minipage width calculation for pandoc >= 2.11.1
    \usepackage[inline]{enumitem} % IRkernel/repr support (it uses the enumerate* environment)
    \usepackage[normalem]{ulem} % ulem is needed to support strikethroughs (\sout)
                                % normalem makes italics be italics, not underlines
    \usepackage{soul}      % strikethrough (\st) support for pandoc >= 3.0.0
    \usepackage{mathrsfs}
    

    
    % Colors for the hyperref package
    \definecolor{urlcolor}{rgb}{0,.145,.698}
    \definecolor{linkcolor}{rgb}{.71,0.21,0.01}
    \definecolor{citecolor}{rgb}{.12,.54,.11}

    % ANSI colors
    \definecolor{ansi-black}{HTML}{3E424D}
    \definecolor{ansi-black-intense}{HTML}{282C36}
    \definecolor{ansi-red}{HTML}{E75C58}
    \definecolor{ansi-red-intense}{HTML}{B22B31}
    \definecolor{ansi-green}{HTML}{00A250}
    \definecolor{ansi-green-intense}{HTML}{007427}
    \definecolor{ansi-yellow}{HTML}{DDB62B}
    \definecolor{ansi-yellow-intense}{HTML}{B27D12}
    \definecolor{ansi-blue}{HTML}{208FFB}
    \definecolor{ansi-blue-intense}{HTML}{0065CA}
    \definecolor{ansi-magenta}{HTML}{D160C4}
    \definecolor{ansi-magenta-intense}{HTML}{A03196}
    \definecolor{ansi-cyan}{HTML}{60C6C8}
    \definecolor{ansi-cyan-intense}{HTML}{258F8F}
    \definecolor{ansi-white}{HTML}{C5C1B4}
    \definecolor{ansi-white-intense}{HTML}{A1A6B2}
    \definecolor{ansi-default-inverse-fg}{HTML}{FFFFFF}
    \definecolor{ansi-default-inverse-bg}{HTML}{000000}

    % common color for the border for error outputs.
    \definecolor{outerrorbackground}{HTML}{FFDFDF}

    % commands and environments needed by pandoc snippets
    % extracted from the output of `pandoc -s`
    \providecommand{\tightlist}{%
      \setlength{\itemsep}{0pt}\setlength{\parskip}{0pt}}
    \DefineVerbatimEnvironment{Highlighting}{Verbatim}{commandchars=\\\{\}}
    % Add ',fontsize=\small' for more characters per line
    \newenvironment{Shaded}{}{}
    \newcommand{\KeywordTok}[1]{\textcolor[rgb]{0.00,0.44,0.13}{\textbf{{#1}}}}
    \newcommand{\DataTypeTok}[1]{\textcolor[rgb]{0.56,0.13,0.00}{{#1}}}
    \newcommand{\DecValTok}[1]{\textcolor[rgb]{0.25,0.63,0.44}{{#1}}}
    \newcommand{\BaseNTok}[1]{\textcolor[rgb]{0.25,0.63,0.44}{{#1}}}
    \newcommand{\FloatTok}[1]{\textcolor[rgb]{0.25,0.63,0.44}{{#1}}}
    \newcommand{\CharTok}[1]{\textcolor[rgb]{0.25,0.44,0.63}{{#1}}}
    \newcommand{\StringTok}[1]{\textcolor[rgb]{0.25,0.44,0.63}{{#1}}}
    \newcommand{\CommentTok}[1]{\textcolor[rgb]{0.38,0.63,0.69}{\textit{{#1}}}}
    \newcommand{\OtherTok}[1]{\textcolor[rgb]{0.00,0.44,0.13}{{#1}}}
    \newcommand{\AlertTok}[1]{\textcolor[rgb]{1.00,0.00,0.00}{\textbf{{#1}}}}
    \newcommand{\FunctionTok}[1]{\textcolor[rgb]{0.02,0.16,0.49}{{#1}}}
    \newcommand{\RegionMarkerTok}[1]{{#1}}
    \newcommand{\ErrorTok}[1]{\textcolor[rgb]{1.00,0.00,0.00}{\textbf{{#1}}}}
    \newcommand{\NormalTok}[1]{{#1}}

    % Additional commands for more recent versions of Pandoc
    \newcommand{\ConstantTok}[1]{\textcolor[rgb]{0.53,0.00,0.00}{{#1}}}
    \newcommand{\SpecialCharTok}[1]{\textcolor[rgb]{0.25,0.44,0.63}{{#1}}}
    \newcommand{\VerbatimStringTok}[1]{\textcolor[rgb]{0.25,0.44,0.63}{{#1}}}
    \newcommand{\SpecialStringTok}[1]{\textcolor[rgb]{0.73,0.40,0.53}{{#1}}}
    \newcommand{\ImportTok}[1]{{#1}}
    \newcommand{\DocumentationTok}[1]{\textcolor[rgb]{0.73,0.13,0.13}{\textit{{#1}}}}
    \newcommand{\AnnotationTok}[1]{\textcolor[rgb]{0.38,0.63,0.69}{\textbf{\textit{{#1}}}}}
    \newcommand{\CommentVarTok}[1]{\textcolor[rgb]{0.38,0.63,0.69}{\textbf{\textit{{#1}}}}}
    \newcommand{\VariableTok}[1]{\textcolor[rgb]{0.10,0.09,0.49}{{#1}}}
    \newcommand{\ControlFlowTok}[1]{\textcolor[rgb]{0.00,0.44,0.13}{\textbf{{#1}}}}
    \newcommand{\OperatorTok}[1]{\textcolor[rgb]{0.40,0.40,0.40}{{#1}}}
    \newcommand{\BuiltInTok}[1]{{#1}}
    \newcommand{\ExtensionTok}[1]{{#1}}
    \newcommand{\PreprocessorTok}[1]{\textcolor[rgb]{0.74,0.48,0.00}{{#1}}}
    \newcommand{\AttributeTok}[1]{\textcolor[rgb]{0.49,0.56,0.16}{{#1}}}
    \newcommand{\InformationTok}[1]{\textcolor[rgb]{0.38,0.63,0.69}{\textbf{\textit{{#1}}}}}
    \newcommand{\WarningTok}[1]{\textcolor[rgb]{0.38,0.63,0.69}{\textbf{\textit{{#1}}}}}
    \makeatletter
    \newsavebox\pandoc@box
    \newcommand*\pandocbounded[1]{%
      \sbox\pandoc@box{#1}%
      % scaling factors for width and height
      \Gscale@div\@tempa\textheight{\dimexpr\ht\pandoc@box+\dp\pandoc@box\relax}%
      \Gscale@div\@tempb\linewidth{\wd\pandoc@box}%
      % select the smaller of both
      \ifdim\@tempb\p@<\@tempa\p@
        \let\@tempa\@tempb
      \fi
      % scaling accordingly (\@tempa < 1)
      \ifdim\@tempa\p@<\p@
        \scalebox{\@tempa}{\usebox\pandoc@box}%
      % scaling not needed, use as it is
      \else
        \usebox{\pandoc@box}%
      \fi
    }
    \makeatother

    % Define a nice break command that doesn't care if a line doesn't already
    % exist.
    \def\br{\hspace*{\fill} \\* }
    % Math Jax compatibility definitions
    \def\gt{>}
    \def\lt{<}
    \let\Oldtex\TeX
    \let\Oldlatex\LaTeX
    \renewcommand{\TeX}{\textrm{\Oldtex}}
    \renewcommand{\LaTeX}{\textrm{\Oldlatex}}
    % Document parameters
    % Document title
    \title{assn1}
    
    
    
    
    
    
    
% Pygments definitions
\makeatletter
\def\PY@reset{\let\PY@it=\relax \let\PY@bf=\relax%
    \let\PY@ul=\relax \let\PY@tc=\relax%
    \let\PY@bc=\relax \let\PY@ff=\relax}
\def\PY@tok#1{\csname PY@tok@#1\endcsname}
\def\PY@toks#1+{\ifx\relax#1\empty\else%
    \PY@tok{#1}\expandafter\PY@toks\fi}
\def\PY@do#1{\PY@bc{\PY@tc{\PY@ul{%
    \PY@it{\PY@bf{\PY@ff{#1}}}}}}}
\def\PY#1#2{\PY@reset\PY@toks#1+\relax+\PY@do{#2}}

\@namedef{PY@tok@w}{\def\PY@tc##1{\textcolor[rgb]{0.73,0.73,0.73}{##1}}}
\@namedef{PY@tok@c}{\let\PY@it=\textit\def\PY@tc##1{\textcolor[rgb]{0.24,0.48,0.48}{##1}}}
\@namedef{PY@tok@cp}{\def\PY@tc##1{\textcolor[rgb]{0.61,0.40,0.00}{##1}}}
\@namedef{PY@tok@k}{\let\PY@bf=\textbf\def\PY@tc##1{\textcolor[rgb]{0.00,0.50,0.00}{##1}}}
\@namedef{PY@tok@kp}{\def\PY@tc##1{\textcolor[rgb]{0.00,0.50,0.00}{##1}}}
\@namedef{PY@tok@kt}{\def\PY@tc##1{\textcolor[rgb]{0.69,0.00,0.25}{##1}}}
\@namedef{PY@tok@o}{\def\PY@tc##1{\textcolor[rgb]{0.40,0.40,0.40}{##1}}}
\@namedef{PY@tok@ow}{\let\PY@bf=\textbf\def\PY@tc##1{\textcolor[rgb]{0.67,0.13,1.00}{##1}}}
\@namedef{PY@tok@nb}{\def\PY@tc##1{\textcolor[rgb]{0.00,0.50,0.00}{##1}}}
\@namedef{PY@tok@nf}{\def\PY@tc##1{\textcolor[rgb]{0.00,0.00,1.00}{##1}}}
\@namedef{PY@tok@nc}{\let\PY@bf=\textbf\def\PY@tc##1{\textcolor[rgb]{0.00,0.00,1.00}{##1}}}
\@namedef{PY@tok@nn}{\let\PY@bf=\textbf\def\PY@tc##1{\textcolor[rgb]{0.00,0.00,1.00}{##1}}}
\@namedef{PY@tok@ne}{\let\PY@bf=\textbf\def\PY@tc##1{\textcolor[rgb]{0.80,0.25,0.22}{##1}}}
\@namedef{PY@tok@nv}{\def\PY@tc##1{\textcolor[rgb]{0.10,0.09,0.49}{##1}}}
\@namedef{PY@tok@no}{\def\PY@tc##1{\textcolor[rgb]{0.53,0.00,0.00}{##1}}}
\@namedef{PY@tok@nl}{\def\PY@tc##1{\textcolor[rgb]{0.46,0.46,0.00}{##1}}}
\@namedef{PY@tok@ni}{\let\PY@bf=\textbf\def\PY@tc##1{\textcolor[rgb]{0.44,0.44,0.44}{##1}}}
\@namedef{PY@tok@na}{\def\PY@tc##1{\textcolor[rgb]{0.41,0.47,0.13}{##1}}}
\@namedef{PY@tok@nt}{\let\PY@bf=\textbf\def\PY@tc##1{\textcolor[rgb]{0.00,0.50,0.00}{##1}}}
\@namedef{PY@tok@nd}{\def\PY@tc##1{\textcolor[rgb]{0.67,0.13,1.00}{##1}}}
\@namedef{PY@tok@s}{\def\PY@tc##1{\textcolor[rgb]{0.73,0.13,0.13}{##1}}}
\@namedef{PY@tok@sd}{\let\PY@it=\textit\def\PY@tc##1{\textcolor[rgb]{0.73,0.13,0.13}{##1}}}
\@namedef{PY@tok@si}{\let\PY@bf=\textbf\def\PY@tc##1{\textcolor[rgb]{0.64,0.35,0.47}{##1}}}
\@namedef{PY@tok@se}{\let\PY@bf=\textbf\def\PY@tc##1{\textcolor[rgb]{0.67,0.36,0.12}{##1}}}
\@namedef{PY@tok@sr}{\def\PY@tc##1{\textcolor[rgb]{0.64,0.35,0.47}{##1}}}
\@namedef{PY@tok@ss}{\def\PY@tc##1{\textcolor[rgb]{0.10,0.09,0.49}{##1}}}
\@namedef{PY@tok@sx}{\def\PY@tc##1{\textcolor[rgb]{0.00,0.50,0.00}{##1}}}
\@namedef{PY@tok@m}{\def\PY@tc##1{\textcolor[rgb]{0.40,0.40,0.40}{##1}}}
\@namedef{PY@tok@gh}{\let\PY@bf=\textbf\def\PY@tc##1{\textcolor[rgb]{0.00,0.00,0.50}{##1}}}
\@namedef{PY@tok@gu}{\let\PY@bf=\textbf\def\PY@tc##1{\textcolor[rgb]{0.50,0.00,0.50}{##1}}}
\@namedef{PY@tok@gd}{\def\PY@tc##1{\textcolor[rgb]{0.63,0.00,0.00}{##1}}}
\@namedef{PY@tok@gi}{\def\PY@tc##1{\textcolor[rgb]{0.00,0.52,0.00}{##1}}}
\@namedef{PY@tok@gr}{\def\PY@tc##1{\textcolor[rgb]{0.89,0.00,0.00}{##1}}}
\@namedef{PY@tok@ge}{\let\PY@it=\textit}
\@namedef{PY@tok@gs}{\let\PY@bf=\textbf}
\@namedef{PY@tok@ges}{\let\PY@bf=\textbf\let\PY@it=\textit}
\@namedef{PY@tok@gp}{\let\PY@bf=\textbf\def\PY@tc##1{\textcolor[rgb]{0.00,0.00,0.50}{##1}}}
\@namedef{PY@tok@go}{\def\PY@tc##1{\textcolor[rgb]{0.44,0.44,0.44}{##1}}}
\@namedef{PY@tok@gt}{\def\PY@tc##1{\textcolor[rgb]{0.00,0.27,0.87}{##1}}}
\@namedef{PY@tok@err}{\def\PY@bc##1{{\setlength{\fboxsep}{\string -\fboxrule}\fcolorbox[rgb]{1.00,0.00,0.00}{1,1,1}{\strut ##1}}}}
\@namedef{PY@tok@kc}{\let\PY@bf=\textbf\def\PY@tc##1{\textcolor[rgb]{0.00,0.50,0.00}{##1}}}
\@namedef{PY@tok@kd}{\let\PY@bf=\textbf\def\PY@tc##1{\textcolor[rgb]{0.00,0.50,0.00}{##1}}}
\@namedef{PY@tok@kn}{\let\PY@bf=\textbf\def\PY@tc##1{\textcolor[rgb]{0.00,0.50,0.00}{##1}}}
\@namedef{PY@tok@kr}{\let\PY@bf=\textbf\def\PY@tc##1{\textcolor[rgb]{0.00,0.50,0.00}{##1}}}
\@namedef{PY@tok@bp}{\def\PY@tc##1{\textcolor[rgb]{0.00,0.50,0.00}{##1}}}
\@namedef{PY@tok@fm}{\def\PY@tc##1{\textcolor[rgb]{0.00,0.00,1.00}{##1}}}
\@namedef{PY@tok@vc}{\def\PY@tc##1{\textcolor[rgb]{0.10,0.09,0.49}{##1}}}
\@namedef{PY@tok@vg}{\def\PY@tc##1{\textcolor[rgb]{0.10,0.09,0.49}{##1}}}
\@namedef{PY@tok@vi}{\def\PY@tc##1{\textcolor[rgb]{0.10,0.09,0.49}{##1}}}
\@namedef{PY@tok@vm}{\def\PY@tc##1{\textcolor[rgb]{0.10,0.09,0.49}{##1}}}
\@namedef{PY@tok@sa}{\def\PY@tc##1{\textcolor[rgb]{0.73,0.13,0.13}{##1}}}
\@namedef{PY@tok@sb}{\def\PY@tc##1{\textcolor[rgb]{0.73,0.13,0.13}{##1}}}
\@namedef{PY@tok@sc}{\def\PY@tc##1{\textcolor[rgb]{0.73,0.13,0.13}{##1}}}
\@namedef{PY@tok@dl}{\def\PY@tc##1{\textcolor[rgb]{0.73,0.13,0.13}{##1}}}
\@namedef{PY@tok@s2}{\def\PY@tc##1{\textcolor[rgb]{0.73,0.13,0.13}{##1}}}
\@namedef{PY@tok@sh}{\def\PY@tc##1{\textcolor[rgb]{0.73,0.13,0.13}{##1}}}
\@namedef{PY@tok@s1}{\def\PY@tc##1{\textcolor[rgb]{0.73,0.13,0.13}{##1}}}
\@namedef{PY@tok@mb}{\def\PY@tc##1{\textcolor[rgb]{0.40,0.40,0.40}{##1}}}
\@namedef{PY@tok@mf}{\def\PY@tc##1{\textcolor[rgb]{0.40,0.40,0.40}{##1}}}
\@namedef{PY@tok@mh}{\def\PY@tc##1{\textcolor[rgb]{0.40,0.40,0.40}{##1}}}
\@namedef{PY@tok@mi}{\def\PY@tc##1{\textcolor[rgb]{0.40,0.40,0.40}{##1}}}
\@namedef{PY@tok@il}{\def\PY@tc##1{\textcolor[rgb]{0.40,0.40,0.40}{##1}}}
\@namedef{PY@tok@mo}{\def\PY@tc##1{\textcolor[rgb]{0.40,0.40,0.40}{##1}}}
\@namedef{PY@tok@ch}{\let\PY@it=\textit\def\PY@tc##1{\textcolor[rgb]{0.24,0.48,0.48}{##1}}}
\@namedef{PY@tok@cm}{\let\PY@it=\textit\def\PY@tc##1{\textcolor[rgb]{0.24,0.48,0.48}{##1}}}
\@namedef{PY@tok@cpf}{\let\PY@it=\textit\def\PY@tc##1{\textcolor[rgb]{0.24,0.48,0.48}{##1}}}
\@namedef{PY@tok@c1}{\let\PY@it=\textit\def\PY@tc##1{\textcolor[rgb]{0.24,0.48,0.48}{##1}}}
\@namedef{PY@tok@cs}{\let\PY@it=\textit\def\PY@tc##1{\textcolor[rgb]{0.24,0.48,0.48}{##1}}}

\def\PYZbs{\char`\\}
\def\PYZus{\char`\_}
\def\PYZob{\char`\{}
\def\PYZcb{\char`\}}
\def\PYZca{\char`\^}
\def\PYZam{\char`\&}
\def\PYZlt{\char`\<}
\def\PYZgt{\char`\>}
\def\PYZsh{\char`\#}
\def\PYZpc{\char`\%}
\def\PYZdl{\char`\$}
\def\PYZhy{\char`\-}
\def\PYZsq{\char`\'}
\def\PYZdq{\char`\"}
\def\PYZti{\char`\~}
% for compatibility with earlier versions
\def\PYZat{@}
\def\PYZlb{[}
\def\PYZrb{]}
\makeatother


    % For linebreaks inside Verbatim environment from package fancyvrb.
    \makeatletter
        \newbox\Wrappedcontinuationbox
        \newbox\Wrappedvisiblespacebox
        \newcommand*\Wrappedvisiblespace {\textcolor{red}{\textvisiblespace}}
        \newcommand*\Wrappedcontinuationsymbol {\textcolor{red}{\llap{\tiny$\m@th\hookrightarrow$}}}
        \newcommand*\Wrappedcontinuationindent {3ex }
        \newcommand*\Wrappedafterbreak {\kern\Wrappedcontinuationindent\copy\Wrappedcontinuationbox}
        % Take advantage of the already applied Pygments mark-up to insert
        % potential linebreaks for TeX processing.
        %        {, <, #, %, $, ' and ": go to next line.
        %        _, }, ^, &, >, - and ~: stay at end of broken line.
        % Use of \textquotesingle for straight quote.
        \newcommand*\Wrappedbreaksatspecials {%
            \def\PYGZus{\discretionary{\char`\_}{\Wrappedafterbreak}{\char`\_}}%
            \def\PYGZob{\discretionary{}{\Wrappedafterbreak\char`\{}{\char`\{}}%
            \def\PYGZcb{\discretionary{\char`\}}{\Wrappedafterbreak}{\char`\}}}%
            \def\PYGZca{\discretionary{\char`\^}{\Wrappedafterbreak}{\char`\^}}%
            \def\PYGZam{\discretionary{\char`\&}{\Wrappedafterbreak}{\char`\&}}%
            \def\PYGZlt{\discretionary{}{\Wrappedafterbreak\char`\<}{\char`\<}}%
            \def\PYGZgt{\discretionary{\char`\>}{\Wrappedafterbreak}{\char`\>}}%
            \def\PYGZsh{\discretionary{}{\Wrappedafterbreak\char`\#}{\char`\#}}%
            \def\PYGZpc{\discretionary{}{\Wrappedafterbreak\char`\%}{\char`\%}}%
            \def\PYGZdl{\discretionary{}{\Wrappedafterbreak\char`\$}{\char`\$}}%
            \def\PYGZhy{\discretionary{\char`\-}{\Wrappedafterbreak}{\char`\-}}%
            \def\PYGZsq{\discretionary{}{\Wrappedafterbreak\textquotesingle}{\textquotesingle}}%
            \def\PYGZdq{\discretionary{}{\Wrappedafterbreak\char`\"}{\char`\"}}%
            \def\PYGZti{\discretionary{\char`\~}{\Wrappedafterbreak}{\char`\~}}%
        }
        % Some characters . , ; ? ! / are not pygmentized.
        % This macro makes them "active" and they will insert potential linebreaks
        \newcommand*\Wrappedbreaksatpunct {%
            \lccode`\~`\.\lowercase{\def~}{\discretionary{\hbox{\char`\.}}{\Wrappedafterbreak}{\hbox{\char`\.}}}%
            \lccode`\~`\,\lowercase{\def~}{\discretionary{\hbox{\char`\,}}{\Wrappedafterbreak}{\hbox{\char`\,}}}%
            \lccode`\~`\;\lowercase{\def~}{\discretionary{\hbox{\char`\;}}{\Wrappedafterbreak}{\hbox{\char`\;}}}%
            \lccode`\~`\:\lowercase{\def~}{\discretionary{\hbox{\char`\:}}{\Wrappedafterbreak}{\hbox{\char`\:}}}%
            \lccode`\~`\?\lowercase{\def~}{\discretionary{\hbox{\char`\?}}{\Wrappedafterbreak}{\hbox{\char`\?}}}%
            \lccode`\~`\!\lowercase{\def~}{\discretionary{\hbox{\char`\!}}{\Wrappedafterbreak}{\hbox{\char`\!}}}%
            \lccode`\~`\/\lowercase{\def~}{\discretionary{\hbox{\char`\/}}{\Wrappedafterbreak}{\hbox{\char`\/}}}%
            \catcode`\.\active
            \catcode`\,\active
            \catcode`\;\active
            \catcode`\:\active
            \catcode`\?\active
            \catcode`\!\active
            \catcode`\/\active
            \lccode`\~`\~
        }
    \makeatother

    \let\OriginalVerbatim=\Verbatim
    \makeatletter
    \renewcommand{\Verbatim}[1][1]{%
        %\parskip\z@skip
        \sbox\Wrappedcontinuationbox {\Wrappedcontinuationsymbol}%
        \sbox\Wrappedvisiblespacebox {\FV@SetupFont\Wrappedvisiblespace}%
        \def\FancyVerbFormatLine ##1{\hsize\linewidth
            \vtop{\raggedright\hyphenpenalty\z@\exhyphenpenalty\z@
                \doublehyphendemerits\z@\finalhyphendemerits\z@
                \strut ##1\strut}%
        }%
        % If the linebreak is at a space, the latter will be displayed as visible
        % space at end of first line, and a continuation symbol starts next line.
        % Stretch/shrink are however usually zero for typewriter font.
        \def\FV@Space {%
            \nobreak\hskip\z@ plus\fontdimen3\font minus\fontdimen4\font
            \discretionary{\copy\Wrappedvisiblespacebox}{\Wrappedafterbreak}
            {\kern\fontdimen2\font}%
        }%

        % Allow breaks at special characters using \PYG... macros.
        \Wrappedbreaksatspecials
        % Breaks at punctuation characters . , ; ? ! and / need catcode=\active
        \OriginalVerbatim[#1,codes*=\Wrappedbreaksatpunct]%
    }
    \makeatother

    % Exact colors from NB
    \definecolor{incolor}{HTML}{303F9F}
    \definecolor{outcolor}{HTML}{D84315}
    \definecolor{cellborder}{HTML}{CFCFCF}
    \definecolor{cellbackground}{HTML}{F7F7F7}

    % prompt
    \makeatletter
    \newcommand{\boxspacing}{\kern\kvtcb@left@rule\kern\kvtcb@boxsep}
    \makeatother
    \newcommand{\prompt}[4]{
        {\ttfamily\llap{{\color{#2}[#3]:\hspace{3pt}#4}}\vspace{-\baselineskip}}
    }
    

    
    % Prevent overflowing lines due to hard-to-break entities
    \sloppy
    % Setup hyperref package
    \hypersetup{
      breaklinks=true,  % so long urls are correctly broken across lines
      colorlinks=true,
      urlcolor=urlcolor,
      linkcolor=linkcolor,
      citecolor=citecolor,
      }
    % Slightly bigger margins than the latex defaults
    
    \geometry{verbose,tmargin=1in,bmargin=1in,lmargin=1in,rmargin=1in}
    
    

\begin{document}
    
    \maketitle
    
    

    
    \section{Environment}\label{environment}

    \subsection{\texorpdfstring{Rust and
\texttt{evcxr\_jupyter}}{Rust and evcxr\_jupyter}}\label{rust-and-evcxr_jupyter}

    This notebook is written in Rust using the \texttt{evcxr} kernel. In
order to install this environment, follow the instructions
\href{https://github.com/evcxr/evcxr/blob/main/evcxr_jupyter/README.md\#installation}{here}
before trying to run this notebook.

    \subsection{Report Generation}\label{report-generation}

    In the report form of this notebook, I use some utility functions for
\(\LaTeX\)-friendly formatting. I include these in an
\texttt{init.evcxr} and \texttt{prelude.rs} file. These should be placed
in the \texttt{evcxr} config directory as described
\href{https://github.com/evcxr/evcxr/blob/main/COMMON.md\#startup}{here}
before starting up the notebook.

If you're running the notebook as code and not trying to generate a
report, this is unnecessary.

    \begin{tcolorbox}[breakable, size=fbox, boxrule=1pt, pad at break*=1mm,colback=cellbackground, colframe=cellborder]
\prompt{In}{incolor}{2}{\boxspacing}
\begin{Verbatim}[commandchars=\\\{\}]
\PY{p}{:}\PY{n+nc}{fmt}\PY{+w}{ }\PY{p}{\PYZob{}}\PY{p}{\PYZcb{}}
\end{Verbatim}
\end{tcolorbox}

            \begin{tcolorbox}[breakable, size=fbox, boxrule=.5pt, pad at break*=1mm, opacityfill=0]
\prompt{Out}{outcolor}{2}{\boxspacing}
\begin{Verbatim}[commandchars=\\\{\}]
Output format: \{\}

\end{Verbatim}
\end{tcolorbox}
        
    \subsection{Imports}\label{imports}

    \begin{tcolorbox}[breakable, size=fbox, boxrule=1pt, pad at break*=1mm,colback=cellbackground, colframe=cellborder]
\prompt{In}{incolor}{3}{\boxspacing}
\begin{Verbatim}[commandchars=\\\{\}]
\PY{p}{:}\PY{n+nc}{dep}\PY{+w}{ }\PY{n}{polars}\PY{+w}{ }\PY{o}{=}\PY{+w}{ }\PY{p}{\PYZob{}}\PY{+w}{ }\PY{n}{version}\PY{+w}{ }\PY{o}{=}\PY{+w}{ }\PY{l+s}{\PYZdq{}}\PY{l+s}{0.46}\PY{l+s}{\PYZdq{}}\PY{p}{,}\PY{+w}{ }\PY{n}{features}\PY{+w}{ }\PY{o}{=}\PY{+w}{ }\PY{p}{[}\PY{+w}{ }\PY{l+s}{\PYZdq{}}\PY{l+s}{lazy}\PY{l+s}{\PYZdq{}}\PY{p}{,}\PY{+w}{ }\PY{l+s}{\PYZdq{}}\PY{l+s}{list\PYZus{}arithmetic}\PY{l+s}{\PYZdq{}}\PY{p}{,}\PY{+w}{ }\PY{l+s}{\PYZdq{}}\PY{l+s}{round\PYZus{}series}\PY{l+s}{\PYZdq{}}\PY{p}{,}\PY{+w}{ }\PY{l+s}{\PYZdq{}}\PY{l+s}{log}\PY{l+s}{\PYZdq{}}\PY{p}{,}\PY{+w}{ }\PY{l+s}{\PYZdq{}}\PY{l+s}{range}\PY{l+s}{\PYZdq{}}\PY{p}{]}\PY{+w}{ }\PY{p}{\PYZcb{}}
\PY{p}{:}\PY{n+nc}{dep}\PY{+w}{ }\PY{n}{nalgebra}
\end{Verbatim}
\end{tcolorbox}

    \begin{tcolorbox}[breakable, size=fbox, boxrule=1pt, pad at break*=1mm,colback=cellbackground, colframe=cellborder]
\prompt{In}{incolor}{4}{\boxspacing}
\begin{Verbatim}[commandchars=\\\{\}]
\PY{c+cp}{\PYZsh{}![}\PY{c+cp}{allow(non\PYZus{}snake\PYZus{}case)}\PY{c+cp}{]}
\PY{k}{use}\PY{+w}{ }\PY{n}{polars}\PY{p}{::}\PY{n}{prelude}\PY{p}{:}\PY{p}{:}\PY{o}{*}\PY{p}{;}
\PY{k}{use}\PY{+w}{ }\PY{n}{std}\PY{p}{::}\PY{k+kt}{f64}\PY{p}{::}\PY{n}{consts}\PY{p}{::}\PY{n}{E}\PY{p}{;}
\end{Verbatim}
\end{tcolorbox}

    \section{Utility functions}\label{utility-functions}

    \subsection{Log Utility}\label{log-utility}

    \subsubsection{\texorpdfstring{(a) \textbf{Certainty Equivalent and Risk
Premium}}{(a) Certainty Equivalent and Risk Premium}}\label{a-certainty-equivalent-and-risk-premium}

    Initialize data. Our inputs are: - \(W_0\) = wealth - \(p_w\) =
probability of winning - \(w\) = amount won - \(p_l\) = probability of
losing - \(l\) = amount lost

where the risk \(z\) represents

\[z = \begin{cases} w, & p_w \\ l, & p_l \end{cases}\]

    \begin{tcolorbox}[breakable, size=fbox, boxrule=1pt, pad at break*=1mm,colback=cellbackground, colframe=cellborder]
\prompt{In}{incolor}{10}{\boxspacing}
\begin{Verbatim}[commandchars=\\\{\}]
\PY{k+kd}{let}\PY{+w}{ }\PY{k}{mut}\PY{+w}{ }\PY{n}{df}\PY{+w}{ }\PY{o}{=}\PY{+w}{ }\PY{n}{df}\PY{o}{!}\PY{p}{(}
\PY{+w}{    }\PY{l+s}{\PYZdq{}}\PY{l+s}{W\PYZus{}0}\PY{l+s}{\PYZdq{}}\PY{+w}{ }\PY{o}{=}\PY{o}{\PYZgt{}}\PY{+w}{ }\PY{p}{[}\PY{l+m+mf}{1500.0}\PY{p}{]}\PY{p}{,}
\PY{+w}{    }\PY{l+s}{\PYZdq{}}\PY{l+s}{p\PYZus{}w}\PY{l+s}{\PYZdq{}}\PY{+w}{ }\PY{o}{=}\PY{o}{\PYZgt{}}\PY{+w}{ }\PY{p}{[}\PY{l+m+mf}{0.5}\PY{p}{]}\PY{p}{,}
\PY{+w}{    }\PY{l+s}{\PYZdq{}}\PY{l+s}{p\PYZus{}l}\PY{l+s}{\PYZdq{}}\PY{+w}{ }\PY{o}{=}\PY{o}{\PYZgt{}}\PY{+w}{ }\PY{p}{[}\PY{l+m+mf}{0.5}\PY{p}{]}\PY{p}{,}
\PY{+w}{    }\PY{l+s}{\PYZdq{}}\PY{l+s}{w}\PY{l+s}{\PYZdq{}}\PY{+w}{ }\PY{o}{=}\PY{o}{\PYZgt{}}\PY{+w}{ }\PY{p}{[}\PY{l+m+mf}{150.0}\PY{p}{]}\PY{p}{,}
\PY{+w}{    }\PY{l+s}{\PYZdq{}}\PY{l+s}{l}\PY{l+s}{\PYZdq{}}\PY{+w}{ }\PY{o}{=}\PY{o}{\PYZgt{}}\PY{+w}{ }\PY{p}{[}\PY{l+m+mf}{150.0}\PY{p}{]}
\PY{p}{)}\PY{o}{?}\PY{p}{;}
\PY{k+kd}{let}\PY{+w}{ }\PY{n}{W\PYZus{}0}\PY{+w}{ }\PY{o}{=}\PY{+w}{ }\PY{n}{col}\PY{p}{(}\PY{l+s}{\PYZdq{}}\PY{l+s}{W\PYZus{}0}\PY{l+s}{\PYZdq{}}\PY{p}{)}\PY{p}{;}
\PY{k+kd}{let}\PY{+w}{ }\PY{n}{p\PYZus{}w}\PY{+w}{ }\PY{o}{=}\PY{+w}{ }\PY{n}{col}\PY{p}{(}\PY{l+s}{\PYZdq{}}\PY{l+s}{p\PYZus{}w}\PY{l+s}{\PYZdq{}}\PY{p}{)}\PY{p}{;}
\PY{k+kd}{let}\PY{+w}{ }\PY{n}{p\PYZus{}l}\PY{+w}{ }\PY{o}{=}\PY{+w}{ }\PY{n}{col}\PY{p}{(}\PY{l+s}{\PYZdq{}}\PY{l+s}{p\PYZus{}l}\PY{l+s}{\PYZdq{}}\PY{p}{)}\PY{p}{;}
\PY{k+kd}{let}\PY{+w}{ }\PY{n}{w}\PY{+w}{ }\PY{o}{=}\PY{+w}{ }\PY{n}{col}\PY{p}{(}\PY{l+s}{\PYZdq{}}\PY{l+s}{w}\PY{l+s}{\PYZdq{}}\PY{p}{)}\PY{p}{;}
\PY{k+kd}{let}\PY{+w}{ }\PY{n}{l}\PY{+w}{ }\PY{o}{=}\PY{+w}{ }\PY{n}{col}\PY{p}{(}\PY{l+s}{\PYZdq{}}\PY{l+s}{l}\PY{l+s}{\PYZdq{}}\PY{p}{)}\PY{p}{;}

\PY{n}{LaTeXDataFrame}\PY{p}{(}\PY{n}{df}\PY{p}{.}\PY{n}{clone}\PY{p}{(}\PY{p}{)}\PY{p}{)}\PY{p}{.}\PY{n}{evcxr\PYZus{}display}\PY{p}{(}\PY{p}{)}\PY{p}{;}
\end{Verbatim}
\end{tcolorbox}
 
            
\prompt{Out}{outcolor}{10}{}
    
    \begin{tabularx}{\textwidth}{|*5{p{\dimexpr\textwidth/5-2\tabcolsep\relax}|}}
\hlineW\_0 & p\_w & p\_l & w & l \\
\hline
1500.0000 & 0.5000 & 0.5000 & 150.0000 & 150.0000 \\
\hline
\end{tabularx}

    

    Our utility function \(U\) is the \(\log\) function.

    \begin{tcolorbox}[breakable, size=fbox, boxrule=1pt, pad at break*=1mm,colback=cellbackground, colframe=cellborder]
\prompt{In}{incolor}{11}{\boxspacing}
\begin{Verbatim}[commandchars=\\\{\}]
\PY{k}{pub}\PY{+w}{ }\PY{k}{fn}\PY{+w}{ }\PY{n+nf}{U}\PY{p}{(}\PY{n}{e}\PY{p}{:}\PY{+w}{ }\PY{n+nc}{Expr}\PY{p}{)}\PY{+w}{ }\PY{p}{\PYZhy{}\PYZgt{}}\PY{+w}{ }\PY{n+nc}{Expr}\PY{+w}{ }\PY{p}{\PYZob{}}
\PY{+w}{    }\PY{n}{e}\PY{p}{.}\PY{n}{log}\PY{p}{(}\PY{n}{E}\PY{p}{)}
\PY{p}{\PYZcb{}}

\PY{k}{pub}\PY{+w}{ }\PY{k}{fn}\PY{+w}{ }\PY{n+nf}{U\PYZus{}inv}\PY{p}{(}\PY{n}{e}\PY{p}{:}\PY{+w}{ }\PY{n+nc}{Expr}\PY{p}{)}\PY{+w}{ }\PY{p}{\PYZhy{}\PYZgt{}}\PY{+w}{ }\PY{n+nc}{Expr}\PY{+w}{ }\PY{p}{\PYZob{}}
\PY{+w}{    }\PY{n}{e}\PY{p}{.}\PY{n}{exp}\PY{p}{(}\PY{p}{)}
\PY{p}{\PYZcb{}}
\end{Verbatim}
\end{tcolorbox}

    We can compute the expectation of \(U\) as:

\[\mathbb{E}[U(W_T)] = p_w \cdot U(W_0 + w) + p_l \cdot U(W_0 - l)\]

    \begin{tcolorbox}[breakable, size=fbox, boxrule=1pt, pad at break*=1mm,colback=cellbackground, colframe=cellborder]
\prompt{In}{incolor}{12}{\boxspacing}
\begin{Verbatim}[commandchars=\\\{\}]
\PY{k+kd}{let}\PY{+w}{ }\PY{n}{EUW\PYZus{}T}\PY{+w}{ }\PY{o}{=}\PY{+w}{ }\PY{p}{(}\PY{n}{p\PYZus{}w}\PY{p}{.}\PY{n}{clone}\PY{p}{(}\PY{p}{)}
\PY{+w}{    }\PY{o}{*}\PY{+w}{ }\PY{n}{U}\PY{p}{(}\PY{n}{W\PYZus{}0}\PY{p}{.}\PY{n}{clone}\PY{p}{(}\PY{p}{)}\PY{+w}{ }\PY{o}{+}\PY{+w}{ }\PY{n}{w}\PY{p}{.}\PY{n}{clone}\PY{p}{(}\PY{p}{)}\PY{p}{)}
\PY{+w}{    }\PY{o}{+}\PY{+w}{ }\PY{n}{p\PYZus{}l}\PY{p}{.}\PY{n}{clone}\PY{p}{(}\PY{p}{)}\PY{+w}{ }\PY{o}{*}\PY{+w}{ }\PY{n}{U}\PY{p}{(}\PY{n}{W\PYZus{}0}\PY{p}{.}\PY{n}{clone}\PY{p}{(}\PY{p}{)}\PY{+w}{ }\PY{o}{\PYZhy{}}\PY{+w}{ }\PY{n}{l}\PY{p}{.}\PY{n}{clone}\PY{p}{(}\PY{p}{)}\PY{p}{)}\PY{p}{)}
\PY{+w}{    }\PY{p}{.}\PY{n}{alias}\PY{p}{(}\PY{l+s}{\PYZdq{}}\PY{l+s+se}{\PYZbs{}\PYZbs{}}\PY{l+s}{mathbb\PYZob{}E\PYZcb{}[U(W\PYZus{}T)]}\PY{l+s}{\PYZdq{}}\PY{p}{)}\PY{p}{;}

\PY{n}{LaTeXDataFrame}\PY{p}{(}\PY{n}{df}\PY{p}{.}\PY{n}{clone}\PY{p}{(}\PY{p}{)}\PY{p}{.}\PY{n}{lazy}\PY{p}{(}\PY{p}{)}\PY{p}{.}\PY{n}{select}\PY{p}{(}\PY{p}{[}\PY{n}{EUW\PYZus{}T}\PY{p}{.}\PY{n}{clone}\PY{p}{(}\PY{p}{)}\PY{p}{]}\PY{p}{)}\PY{p}{.}\PY{n}{collect}\PY{p}{(}\PY{p}{)}\PY{o}{?}\PY{p}{)}\PY{p}{.}\PY{n}{evcxr\PYZus{}display}\PY{p}{(}\PY{p}{)}\PY{p}{;}
\end{Verbatim}
\end{tcolorbox}
 
            
\prompt{Out}{outcolor}{12}{}
    
    \begin{tabularx}{\textwidth}{|*1{p{\dimexpr\textwidth/1-2\tabcolsep\relax}|}}
\hline\mathbb{E}[U(W\_T)] \\
\hline
7.3082 \\
\hline
\end{tabularx}

    

    The Certainty Equivalent \(CE\) is given by:
\[\mathbb{E}[U(W_T)] = U(CE)\]

which we can therefore calculate as
\[CE = U^{-1}\left(\mathbb{E}[U(W_T)]\right)\]

    \begin{tcolorbox}[breakable, size=fbox, boxrule=1pt, pad at break*=1mm,colback=cellbackground, colframe=cellborder]
\prompt{In}{incolor}{13}{\boxspacing}
\begin{Verbatim}[commandchars=\\\{\}]
\PY{k+kd}{let}\PY{+w}{ }\PY{n}{CE}\PY{+w}{ }\PY{o}{=}\PY{+w}{ }\PY{p}{(}\PY{n}{U\PYZus{}inv}\PY{p}{(}\PY{n}{EUW\PYZus{}T}\PY{p}{.}\PY{n}{clone}\PY{p}{(}\PY{p}{)}\PY{p}{)}\PY{p}{)}\PY{p}{.}\PY{n}{alias}\PY{p}{(}\PY{l+s}{\PYZdq{}}\PY{l+s}{CE}\PY{l+s}{\PYZdq{}}\PY{p}{)}\PY{p}{;}

\PY{n}{LaTeXDataFrame}\PY{p}{(}\PY{n}{df}\PY{p}{.}\PY{n}{clone}\PY{p}{(}\PY{p}{)}\PY{p}{.}\PY{n}{lazy}\PY{p}{(}\PY{p}{)}\PY{p}{.}\PY{n}{select}\PY{p}{(}\PY{p}{[}\PY{n}{CE}\PY{p}{.}\PY{n}{clone}\PY{p}{(}\PY{p}{)}\PY{p}{]}\PY{p}{)}\PY{p}{.}\PY{n}{collect}\PY{p}{(}\PY{p}{)}\PY{o}{?}\PY{p}{)}\PY{p}{.}\PY{n}{evcxr\PYZus{}display}\PY{p}{(}\PY{p}{)}\PY{p}{;}
\end{Verbatim}
\end{tcolorbox}
 
            
\prompt{Out}{outcolor}{13}{}
    
    \begin{tabularx}{\textwidth}{|*1{p{\dimexpr\textwidth/1-2\tabcolsep\relax}|}}
\hlineCE \\
\hline
1492.4812 \\
\hline
\end{tabularx}

    

    In order to calculate risk premium, we will follow the formula:

\[RP = \mathbb{E}[W_T] - CE\]

    \begin{tcolorbox}[breakable, size=fbox, boxrule=1pt, pad at break*=1mm,colback=cellbackground, colframe=cellborder]
\prompt{In}{incolor}{14}{\boxspacing}
\begin{Verbatim}[commandchars=\\\{\}]
\PY{k+kd}{let}\PY{+w}{ }\PY{n}{EW\PYZus{}T}\PY{+w}{ }\PY{o}{=}\PY{+w}{ }\PY{p}{(}\PY{n}{p\PYZus{}w}\PY{p}{.}\PY{n}{clone}\PY{p}{(}\PY{p}{)}
\PY{+w}{    }\PY{o}{*}\PY{+w}{ }\PY{p}{(}\PY{n}{W\PYZus{}0}\PY{p}{.}\PY{n}{clone}\PY{p}{(}\PY{p}{)}\PY{+w}{ }\PY{o}{+}\PY{+w}{ }\PY{n}{w}\PY{p}{.}\PY{n}{clone}\PY{p}{(}\PY{p}{)}\PY{p}{)}
\PY{+w}{    }\PY{o}{+}\PY{+w}{ }\PY{n}{p\PYZus{}l}\PY{p}{.}\PY{n}{clone}\PY{p}{(}\PY{p}{)}\PY{+w}{ }\PY{o}{*}\PY{+w}{ }\PY{p}{(}\PY{n}{W\PYZus{}0}\PY{p}{.}\PY{n}{clone}\PY{p}{(}\PY{p}{)}\PY{+w}{ }\PY{o}{\PYZhy{}}\PY{+w}{ }\PY{n}{l}\PY{p}{.}\PY{n}{clone}\PY{p}{(}\PY{p}{)}\PY{p}{)}\PY{p}{)}
\PY{+w}{    }\PY{p}{.}\PY{n}{alias}\PY{p}{(}\PY{l+s}{\PYZdq{}}\PY{l+s+se}{\PYZbs{}\PYZbs{}}\PY{l+s}{mathbb\PYZob{}E\PYZcb{}[W\PYZus{}T]}\PY{l+s}{\PYZdq{}}\PY{p}{)}\PY{p}{;}
\PY{k+kd}{let}\PY{+w}{ }\PY{n}{RP}\PY{+w}{ }\PY{o}{=}\PY{+w}{ }\PY{p}{(}\PY{n}{EW\PYZus{}T}\PY{p}{.}\PY{n}{clone}\PY{p}{(}\PY{p}{)}\PY{+w}{ }\PY{o}{\PYZhy{}}\PY{+w}{ }\PY{n}{CE}\PY{p}{.}\PY{n}{clone}\PY{p}{(}\PY{p}{)}\PY{p}{)}\PY{p}{.}\PY{n}{alias}\PY{p}{(}\PY{l+s}{\PYZdq{}}\PY{l+s}{RP}\PY{l+s}{\PYZdq{}}\PY{p}{)}\PY{p}{;}

\PY{n}{LaTeXDataFrame}\PY{p}{(}\PY{n}{df}\PY{p}{.}\PY{n}{clone}\PY{p}{(}\PY{p}{)}\PY{p}{.}\PY{n}{lazy}\PY{p}{(}\PY{p}{)}\PY{p}{.}\PY{n}{select}\PY{p}{(}\PY{p}{[}\PY{n}{RP}\PY{p}{.}\PY{n}{clone}\PY{p}{(}\PY{p}{)}\PY{p}{]}\PY{p}{)}\PY{p}{.}\PY{n}{collect}\PY{p}{(}\PY{p}{)}\PY{o}{?}\PY{p}{)}\PY{p}{.}\PY{n}{evcxr\PYZus{}display}\PY{p}{(}\PY{p}{)}\PY{p}{;}
\end{Verbatim}
\end{tcolorbox}
 
            
\prompt{Out}{outcolor}{14}{}
    
    \begin{tabularx}{\textwidth}{|*1{p{\dimexpr\textwidth/1-2\tabcolsep\relax}|}}
\hlineRP \\
\hline
7.5188 \\
\hline
\end{tabularx}

    

    In order to calculate the Taylor Series approximation of the risk
premium, we will consider

\[W_T = W^*_T + x\]

where \(W^*_T\) is deterministic wealth, and \(x\) is a random variable
with characteristics \(\mathbb{E}[x] = 0\) and
\(\mathbb{V}[x] = \sigma_x^2 = \sigma_{W_T}^2\)

\(W^*_T\) is just another way of expressing expected payoff.
\[W^*_T = \mathbb{E}[W_T]\]

    \begin{tcolorbox}[breakable, size=fbox, boxrule=1pt, pad at break*=1mm,colback=cellbackground, colframe=cellborder]
\prompt{In}{incolor}{15}{\boxspacing}
\begin{Verbatim}[commandchars=\\\{\}]
\PY{k+kd}{let}\PY{+w}{ }\PY{n}{W\PYZus{}Tstar}\PY{+w}{ }\PY{o}{=}\PY{+w}{ }\PY{n}{EW\PYZus{}T}\PY{p}{.}\PY{n}{clone}\PY{p}{(}\PY{p}{)}\PY{p}{.}\PY{n}{alias}\PY{p}{(}\PY{l+s}{\PYZdq{}}\PY{l+s}{W\PYZca{}*\PYZus{}T}\PY{l+s}{\PYZdq{}}\PY{p}{)}\PY{p}{;}

\PY{n}{LaTeXDataFrame}\PY{p}{(}\PY{n}{df}\PY{p}{.}\PY{n}{clone}\PY{p}{(}\PY{p}{)}\PY{p}{.}\PY{n}{lazy}\PY{p}{(}\PY{p}{)}\PY{p}{.}\PY{n}{select}\PY{p}{(}\PY{p}{[}\PY{n}{W\PYZus{}Tstar}\PY{p}{.}\PY{n}{clone}\PY{p}{(}\PY{p}{)}\PY{p}{]}\PY{p}{)}\PY{p}{.}\PY{n}{collect}\PY{p}{(}\PY{p}{)}\PY{o}{?}\PY{p}{)}\PY{p}{.}\PY{n}{evcxr\PYZus{}display}\PY{p}{(}\PY{p}{)}\PY{p}{;}
\end{Verbatim}
\end{tcolorbox}
 
            
\prompt{Out}{outcolor}{15}{}
    
    \begin{tabularx}{\textwidth}{|*1{p{\dimexpr\textwidth/1-2\tabcolsep\relax}|}}
\hlineW^*\_T \\
\hline
1500.0000 \\
\hline
\end{tabularx}

    

    Then, in order to solve for the risk premium \(y\), we will use the
formula \[y \approx -\frac{\sigma_x^2}{2}A(W_T^*)\]

where \textbf{Absolute Risk Aversion} \(A\) is defined as

\[A(W_T^*) = \frac{U''(W_T^*)}{U'(W_T^*)}\]

    The derivatives for the utility equation can be found through basic
calculus.

\[U(W_T) = \ln W_T\] \[U'(W_T) = \frac{1}{W_T}\]
\[U''(W_T) = -\frac{1}{W_T^2}\]

Therefore,

\[A(W_T) = -\frac{\tfrac{1}{W_T^2}}{\tfrac{1}{W_T}} = -\frac{1}{W_T}\]

    \begin{tcolorbox}[breakable, size=fbox, boxrule=1pt, pad at break*=1mm,colback=cellbackground, colframe=cellborder]
\prompt{In}{incolor}{17}{\boxspacing}
\begin{Verbatim}[commandchars=\\\{\}]
\PY{k+kd}{let}\PY{+w}{ }\PY{n}{A}\PY{+w}{ }\PY{o}{=}\PY{+w}{ }\PY{p}{(}\PY{o}{\PYZhy{}}\PY{n}{lit}\PY{p}{(}\PY{l+m+mf}{1.0}\PY{p}{)}\PY{+w}{ }\PY{o}{/}\PY{+w}{ }\PY{n}{W\PYZus{}Tstar}\PY{p}{.}\PY{n}{clone}\PY{p}{(}\PY{p}{)}\PY{p}{)}\PY{p}{.}\PY{n}{alias}\PY{p}{(}\PY{l+s}{\PYZdq{}}\PY{l+s}{A(W\PYZus{}T\PYZca{}*)}\PY{l+s}{\PYZdq{}}\PY{p}{)}\PY{p}{;}

\PY{n}{LaTeXDataFrame}\PY{p}{(}\PY{n}{df}\PY{p}{.}\PY{n}{clone}\PY{p}{(}\PY{p}{)}\PY{p}{.}\PY{n}{lazy}\PY{p}{(}\PY{p}{)}\PY{p}{.}\PY{n}{select}\PY{p}{(}\PY{p}{[}\PY{n}{A}\PY{p}{.}\PY{n}{clone}\PY{p}{(}\PY{p}{)}\PY{p}{]}\PY{p}{)}\PY{p}{.}\PY{n}{collect}\PY{p}{(}\PY{p}{)}\PY{o}{?}\PY{p}{)}\PY{p}{.}\PY{n}{evcxr\PYZus{}display}\PY{p}{(}\PY{p}{)}\PY{p}{;}
\end{Verbatim}
\end{tcolorbox}
 
            
\prompt{Out}{outcolor}{17}{}
    
    \begin{tabularx}{\textwidth}{|*1{p{\dimexpr\textwidth/1-2\tabcolsep\relax}|}}
\hlineA(W\_T^*) \\
\hline
-0.0007 \\
\hline
\end{tabularx}

    

    To get volatility, we use the basic equation for variance:

\[\sigma_x^2 = \sigma_z^2 = \mathbb{E}[(z - \mathbb{E}[z])^2]\]

As \(\mathbb{E}[z] = 0\),

\[\sigma_x^2 = p_w \cdot w^2 + p_l \cdot l^2\]

    \begin{tcolorbox}[breakable, size=fbox, boxrule=1pt, pad at break*=1mm,colback=cellbackground, colframe=cellborder]
\prompt{In}{incolor}{18}{\boxspacing}
\begin{Verbatim}[commandchars=\\\{\}]
\PY{k+kd}{let}\PY{+w}{ }\PY{n}{var}\PY{+w}{ }\PY{o}{=}\PY{+w}{ }\PY{p}{(}\PY{n}{p\PYZus{}w}\PY{p}{.}\PY{n}{clone}\PY{p}{(}\PY{p}{)}\PY{+w}{ }\PY{o}{*}\PY{+w}{ }\PY{n}{w}\PY{p}{.}\PY{n}{clone}\PY{p}{(}\PY{p}{)}\PY{p}{.}\PY{n}{pow}\PY{p}{(}\PY{l+m+mi}{2}\PY{p}{)}
\PY{o}{+}\PY{+w}{ }\PY{n}{p\PYZus{}l}\PY{p}{.}\PY{n}{clone}\PY{p}{(}\PY{p}{)}\PY{+w}{ }\PY{o}{*}\PY{+w}{ }\PY{n}{l}\PY{p}{.}\PY{n}{clone}\PY{p}{(}\PY{p}{)}\PY{p}{.}\PY{n}{pow}\PY{p}{(}\PY{l+m+mi}{2}\PY{p}{)}\PY{p}{)}\PY{p}{.}\PY{n}{alias}\PY{p}{(}\PY{l+s}{\PYZdq{}}\PY{l+s+se}{\PYZbs{}\PYZbs{}}\PY{l+s}{sigma\PYZus{}x\PYZca{}2}\PY{l+s}{\PYZdq{}}\PY{p}{)}\PY{p}{;}

\PY{n}{LaTeXDataFrame}\PY{p}{(}\PY{n}{df}\PY{p}{.}\PY{n}{clone}\PY{p}{(}\PY{p}{)}\PY{p}{.}\PY{n}{lazy}\PY{p}{(}\PY{p}{)}\PY{p}{.}\PY{n}{select}\PY{p}{(}\PY{p}{[}\PY{n}{var}\PY{p}{.}\PY{n}{clone}\PY{p}{(}\PY{p}{)}\PY{p}{]}\PY{p}{)}\PY{p}{.}\PY{n}{collect}\PY{p}{(}\PY{p}{)}\PY{o}{?}\PY{p}{)}\PY{p}{.}\PY{n}{evcxr\PYZus{}display}\PY{p}{(}\PY{p}{)}\PY{p}{;}
\end{Verbatim}
\end{tcolorbox}
 
            
\prompt{Out}{outcolor}{18}{}
    
    \begin{tabularx}{\textwidth}{|*1{p{\dimexpr\textwidth/1-2\tabcolsep\relax}|}}
\hline\sigma\_x^2 \\
\hline
22500.0000 \\
\hline
\end{tabularx}

    

    The Taylor Series approximation of the risk premium \(y\) is calculated
thusly:

    \begin{tcolorbox}[breakable, size=fbox, boxrule=1pt, pad at break*=1mm,colback=cellbackground, colframe=cellborder]
\prompt{In}{incolor}{19}{\boxspacing}
\begin{Verbatim}[commandchars=\\\{\}]
\PY{k+kd}{let}\PY{+w}{ }\PY{n}{y}\PY{+w}{ }\PY{o}{=}\PY{+w}{ }\PY{p}{(}\PY{o}{\PYZhy{}}\PY{n}{var}\PY{p}{.}\PY{n}{clone}\PY{p}{(}\PY{p}{)}\PY{+w}{ }\PY{o}{/}\PY{+w}{ }\PY{n}{lit}\PY{p}{(}\PY{l+m+mf}{2.0}\PY{p}{)}\PY{+w}{ }\PY{o}{*}\PY{+w}{ }\PY{n}{A}\PY{p}{.}\PY{n}{clone}\PY{p}{(}\PY{p}{)}\PY{p}{)}\PY{p}{.}\PY{n}{alias}\PY{p}{(}\PY{l+s}{\PYZdq{}}\PY{l+s}{y}\PY{l+s}{\PYZdq{}}\PY{p}{)}\PY{p}{;}

\PY{n}{LaTeXDataFrame}\PY{p}{(}\PY{n}{df}\PY{p}{.}\PY{n}{clone}\PY{p}{(}\PY{p}{)}\PY{p}{.}\PY{n}{lazy}\PY{p}{(}\PY{p}{)}\PY{p}{.}\PY{n}{select}\PY{p}{(}\PY{p}{[}\PY{n}{y}\PY{p}{.}\PY{n}{clone}\PY{p}{(}\PY{p}{)}\PY{p}{]}\PY{p}{)}\PY{p}{.}\PY{n}{collect}\PY{p}{(}\PY{p}{)}\PY{o}{?}\PY{p}{)}\PY{p}{.}\PY{n}{evcxr\PYZus{}display}\PY{p}{(}\PY{p}{)}\PY{p}{;}
\end{Verbatim}
\end{tcolorbox}
 
            
\prompt{Out}{outcolor}{19}{}
    
    \begin{tabularx}{\textwidth}{|*1{p{\dimexpr\textwidth/1-2\tabcolsep\relax}|}}
\hliney \\
\hline
7.5000 \\
\hline
\end{tabularx}

    

    How good is this approximation?

    \begin{tcolorbox}[breakable, size=fbox, boxrule=1pt, pad at break*=1mm,colback=cellbackground, colframe=cellborder]
\prompt{In}{incolor}{20}{\boxspacing}
\begin{Verbatim}[commandchars=\\\{\}]
\PY{k+kd}{let}\PY{+w}{ }\PY{n}{approximation\PYZus{}error}\PY{+w}{ }\PY{o}{=}\PY{+w}{ }\PY{p}{(}\PY{n}{RP}\PY{p}{.}\PY{n}{clone}\PY{p}{(}\PY{p}{)}\PY{+w}{ }\PY{o}{\PYZhy{}}\PY{+w}{ }\PY{n}{y}\PY{p}{.}\PY{n}{clone}\PY{p}{(}\PY{p}{)}\PY{p}{)}\PY{p}{.}\PY{n}{alias}\PY{p}{(}\PY{l+s}{\PYZdq{}}\PY{l+s}{approximation error}\PY{l+s}{\PYZdq{}}\PY{p}{)}\PY{p}{;}

\PY{n}{LaTeXDataFrame}\PY{p}{(}\PY{n}{df}\PY{p}{.}\PY{n}{clone}\PY{p}{(}\PY{p}{)}\PY{p}{.}\PY{n}{lazy}\PY{p}{(}\PY{p}{)}\PY{p}{.}\PY{n}{select}\PY{p}{(}\PY{p}{[}\PY{n}{approximation\PYZus{}error}\PY{p}{.}\PY{n}{clone}\PY{p}{(}\PY{p}{)}\PY{p}{]}\PY{p}{)}\PY{p}{.}\PY{n}{collect}\PY{p}{(}\PY{p}{)}\PY{o}{?}\PY{p}{)}\PY{p}{.}\PY{n}{evcxr\PYZus{}display}\PY{p}{(}\PY{p}{)}\PY{p}{;}
\end{Verbatim}
\end{tcolorbox}
 
            
\prompt{Out}{outcolor}{20}{}
    
    \begin{tabularx}{\textwidth}{|*1{p{\dimexpr\textwidth/1-2\tabcolsep\relax}|}}
\hlineapproximation error \\
\hline
0.0188 \\
\hline
\end{tabularx}

    

    \subsubsection{\texorpdfstring{(b) \textbf{Sensitivity to initial
wealth}}{(b) Sensitivity to initial wealth}}\label{b-sensitivity-to-initial-wealth}

    Let's recalculate these values with an increased initial wealth.

    \begin{tcolorbox}[breakable, size=fbox, boxrule=1pt, pad at break*=1mm,colback=cellbackground, colframe=cellborder]
\prompt{In}{incolor}{21}{\boxspacing}
\begin{Verbatim}[commandchars=\\\{\}]
\PY{n}{df}\PY{p}{.}\PY{n}{with\PYZus{}column}\PY{p}{(}\PY{n}{Column}\PY{p}{::}\PY{n}{new}\PY{p}{(}
\PY{+w}{    }\PY{l+s}{\PYZdq{}}\PY{l+s}{W\PYZus{}0}\PY{l+s}{\PYZdq{}}\PY{p}{.}\PY{n}{into}\PY{p}{(}\PY{p}{)}\PY{p}{,}
\PY{+w}{    }\PY{p}{[}\PY{l+m+mf}{2000.0}\PY{p}{]}\PY{p}{,}
\PY{p}{)}\PY{p}{)}\PY{o}{?}\PY{p}{;}

\PY{n}{LaTeXDataFrame}\PY{p}{(}\PY{n}{df}\PY{p}{.}\PY{n}{clone}\PY{p}{(}\PY{p}{)}\PY{p}{.}\PY{n}{lazy}\PY{p}{(}\PY{p}{)}\PY{p}{.}\PY{n}{select}\PY{p}{(}\PY{p}{[}\PY{n}{RP}\PY{p}{.}\PY{n}{clone}\PY{p}{(}\PY{p}{)}\PY{p}{]}\PY{p}{)}\PY{p}{.}\PY{n}{collect}\PY{p}{(}\PY{p}{)}\PY{o}{?}\PY{p}{)}\PY{p}{.}\PY{n}{evcxr\PYZus{}display}\PY{p}{(}\PY{p}{)}\PY{p}{;}
\end{Verbatim}
\end{tcolorbox}
 
            
\prompt{Out}{outcolor}{21}{}
    
    \begin{tabularx}{\textwidth}{|*1{p{\dimexpr\textwidth/1-2\tabcolsep\relax}|}}
\hlineRP \\
\hline
5.6329 \\
\hline
\end{tabularx}

    

    Because we are risk-averse, our utility function is concave, and
therefore the increase in wealth is inversely proportional to the risk
premium, making it decrease.

    \subsubsection{\texorpdfstring{(c) \textbf{Sensitivity to
volatility}}{(c) Sensitivity to volatility}}\label{c-sensitivity-to-volatility}

    Now we'll change the values of the outcomes.

    \begin{tcolorbox}[breakable, size=fbox, boxrule=1pt, pad at break*=1mm,colback=cellbackground, colframe=cellborder]
\prompt{In}{incolor}{22}{\boxspacing}
\begin{Verbatim}[commandchars=\\\{\}]
\PY{n}{df}\PY{p}{.}\PY{n}{with\PYZus{}column}\PY{p}{(}\PY{n}{Column}\PY{p}{::}\PY{n}{new}\PY{p}{(}
\PY{+w}{    }\PY{l+s}{\PYZdq{}}\PY{l+s}{W\PYZus{}0}\PY{l+s}{\PYZdq{}}\PY{p}{.}\PY{n}{into}\PY{p}{(}\PY{p}{)}\PY{p}{,}
\PY{+w}{    }\PY{p}{[}\PY{l+m+mf}{1000.0}\PY{p}{]}\PY{p}{,}
\PY{p}{)}\PY{p}{)}\PY{o}{?}\PY{p}{;}
\PY{n}{df}\PY{p}{.}\PY{n}{with\PYZus{}column}\PY{p}{(}\PY{n}{Column}\PY{p}{::}\PY{n}{new}\PY{p}{(}
\PY{+w}{    }\PY{l+s}{\PYZdq{}}\PY{l+s}{w}\PY{l+s}{\PYZdq{}}\PY{p}{.}\PY{n}{into}\PY{p}{(}\PY{p}{)}\PY{p}{,}
\PY{+w}{    }\PY{p}{[}\PY{l+m+mf}{300.0}\PY{p}{]}\PY{p}{,}
\PY{p}{)}\PY{p}{)}\PY{o}{?}\PY{p}{;}
\PY{n}{df}\PY{p}{.}\PY{n}{with\PYZus{}column}\PY{p}{(}\PY{n}{Column}\PY{p}{::}\PY{n}{new}\PY{p}{(}
\PY{+w}{    }\PY{l+s}{\PYZdq{}}\PY{l+s}{l}\PY{l+s}{\PYZdq{}}\PY{p}{.}\PY{n}{into}\PY{p}{(}\PY{p}{)}\PY{p}{,}
\PY{+w}{    }\PY{p}{[}\PY{l+m+mf}{300.0}\PY{p}{]}\PY{p}{,}
\PY{p}{)}\PY{p}{)}\PY{o}{?}\PY{p}{;}

\PY{n}{LaTeXDataFrame}\PY{p}{(}\PY{n}{df}\PY{p}{.}\PY{n}{clone}\PY{p}{(}\PY{p}{)}\PY{p}{.}\PY{n}{lazy}\PY{p}{(}\PY{p}{)}\PY{p}{.}\PY{n}{select}\PY{p}{(}\PY{p}{[}\PY{n}{var}\PY{p}{.}\PY{n}{clone}\PY{p}{(}\PY{p}{)}\PY{p}{,}\PY{+w}{ }\PY{n}{RP}\PY{p}{.}\PY{n}{clone}\PY{p}{(}\PY{p}{)}\PY{p}{]}\PY{p}{)}\PY{p}{.}\PY{n}{collect}\PY{p}{(}\PY{p}{)}\PY{o}{?}\PY{p}{)}\PY{p}{.}\PY{n}{evcxr\PYZus{}display}\PY{p}{(}\PY{p}{)}\PY{p}{;}
\end{Verbatim}
\end{tcolorbox}
 
            
\prompt{Out}{outcolor}{22}{}
    
    \begin{tabularx}{\textwidth}{|*2{p{\dimexpr\textwidth/2-2\tabcolsep\relax}|}}
\hline\sigma\_x^2 & RP \\
\hline
90000.0000 & 46.0608 \\
\hline
\end{tabularx}

    

    Conversely, the risk premium is directly proportional to volatility and
increases at approximately double its rate.

    \subsection{Certainty Equivalent and Risk Premium for a Power
Utility}\label{certainty-equivalent-and-risk-premium-for-a-power-utility}

    Initialize data. Our inputs are the same variables as before, with an
additional \(k\) for utility.

    \begin{tcolorbox}[breakable, size=fbox, boxrule=1pt, pad at break*=1mm,colback=cellbackground, colframe=cellborder]
\prompt{In}{incolor}{23}{\boxspacing}
\begin{Verbatim}[commandchars=\\\{\}]
\PY{k+kd}{let}\PY{+w}{ }\PY{k}{mut}\PY{+w}{ }\PY{n}{df}\PY{+w}{ }\PY{o}{=}\PY{+w}{ }\PY{n}{df}\PY{o}{!}\PY{p}{(}
\PY{+w}{    }\PY{l+s}{\PYZdq{}}\PY{l+s}{W\PYZus{}0}\PY{l+s}{\PYZdq{}}\PY{+w}{ }\PY{o}{=}\PY{o}{\PYZgt{}}\PY{+w}{ }\PY{p}{[}\PY{l+m+mf}{1000.0}\PY{p}{]}\PY{p}{,}
\PY{+w}{    }\PY{l+s}{\PYZdq{}}\PY{l+s}{p\PYZus{}w}\PY{l+s}{\PYZdq{}}\PY{+w}{ }\PY{o}{=}\PY{o}{\PYZgt{}}\PY{+w}{ }\PY{p}{[}\PY{l+m+mf}{2.0}\PY{o}{/}\PY{l+m+mf}{3.0}\PY{p}{]}\PY{p}{,}
\PY{+w}{    }\PY{l+s}{\PYZdq{}}\PY{l+s}{p\PYZus{}l}\PY{l+s}{\PYZdq{}}\PY{+w}{ }\PY{o}{=}\PY{o}{\PYZgt{}}\PY{+w}{ }\PY{p}{[}\PY{l+m+mf}{1.0}\PY{o}{/}\PY{l+m+mf}{3.0}\PY{p}{]}\PY{p}{,}
\PY{+w}{    }\PY{l+s}{\PYZdq{}}\PY{l+s}{w}\PY{l+s}{\PYZdq{}}\PY{+w}{ }\PY{o}{=}\PY{o}{\PYZgt{}}\PY{+w}{ }\PY{p}{[}\PY{l+m+mf}{205.0}\PY{p}{]}\PY{p}{,}
\PY{+w}{    }\PY{l+s}{\PYZdq{}}\PY{l+s}{l}\PY{l+s}{\PYZdq{}}\PY{+w}{ }\PY{o}{=}\PY{o}{\PYZgt{}}\PY{+w}{ }\PY{p}{[}\PY{l+m+mf}{400.0}\PY{p}{]}\PY{p}{,}
\PY{+w}{    }\PY{l+s}{\PYZdq{}}\PY{l+s}{k}\PY{l+s}{\PYZdq{}}\PY{+w}{ }\PY{o}{=}\PY{o}{\PYZgt{}}\PY{+w}{ }\PY{p}{[}\PY{l+m+mf}{0.5}\PY{p}{]}\PY{p}{,}
\PY{p}{)}\PY{o}{?}\PY{p}{;}
\PY{k+kd}{let}\PY{+w}{ }\PY{n}{W\PYZus{}0}\PY{+w}{ }\PY{o}{=}\PY{+w}{ }\PY{n}{col}\PY{p}{(}\PY{l+s}{\PYZdq{}}\PY{l+s}{W\PYZus{}0}\PY{l+s}{\PYZdq{}}\PY{p}{)}\PY{p}{;}
\PY{k+kd}{let}\PY{+w}{ }\PY{n}{p\PYZus{}w}\PY{+w}{ }\PY{o}{=}\PY{+w}{ }\PY{n}{col}\PY{p}{(}\PY{l+s}{\PYZdq{}}\PY{l+s}{p\PYZus{}w}\PY{l+s}{\PYZdq{}}\PY{p}{)}\PY{p}{;}
\PY{k+kd}{let}\PY{+w}{ }\PY{n}{p\PYZus{}l}\PY{+w}{ }\PY{o}{=}\PY{+w}{ }\PY{n}{col}\PY{p}{(}\PY{l+s}{\PYZdq{}}\PY{l+s}{p\PYZus{}l}\PY{l+s}{\PYZdq{}}\PY{p}{)}\PY{p}{;}
\PY{k+kd}{let}\PY{+w}{ }\PY{n}{w}\PY{+w}{ }\PY{o}{=}\PY{+w}{ }\PY{n}{col}\PY{p}{(}\PY{l+s}{\PYZdq{}}\PY{l+s}{w}\PY{l+s}{\PYZdq{}}\PY{p}{)}\PY{p}{;}
\PY{k+kd}{let}\PY{+w}{ }\PY{n}{l}\PY{+w}{ }\PY{o}{=}\PY{+w}{ }\PY{n}{col}\PY{p}{(}\PY{l+s}{\PYZdq{}}\PY{l+s}{l}\PY{l+s}{\PYZdq{}}\PY{p}{)}\PY{p}{;}
\PY{k+kd}{let}\PY{+w}{ }\PY{n}{k}\PY{+w}{ }\PY{o}{=}\PY{+w}{ }\PY{n}{col}\PY{p}{(}\PY{l+s}{\PYZdq{}}\PY{l+s}{k}\PY{l+s}{\PYZdq{}}\PY{p}{)}\PY{p}{;}

\PY{n}{LaTeXDataFrame}\PY{p}{(}\PY{n}{df}\PY{p}{.}\PY{n}{clone}\PY{p}{(}\PY{p}{)}\PY{p}{)}\PY{p}{.}\PY{n}{evcxr\PYZus{}display}\PY{p}{(}\PY{p}{)}\PY{p}{;}
\end{Verbatim}
\end{tcolorbox}
 
            
\prompt{Out}{outcolor}{23}{}
    
    \begin{tabularx}{\textwidth}{|*6{p{\dimexpr\textwidth/6-2\tabcolsep\relax}|}}
\hlineW\_0 & p\_w & p\_l & w & l & k \\
\hline
1000.0000 & 0.6667 & 0.3333 & 205.0000 & 400.0000 & 0.5000 \\
\hline
\end{tabularx}

    

    Our utility function is given by

\[U(W) = W^k\]

    \begin{tcolorbox}[breakable, size=fbox, boxrule=1pt, pad at break*=1mm,colback=cellbackground, colframe=cellborder]
\prompt{In}{incolor}{24}{\boxspacing}
\begin{Verbatim}[commandchars=\\\{\}]
\PY{k}{pub}\PY{+w}{ }\PY{k}{fn}\PY{+w}{ }\PY{n+nf}{U}\PY{p}{(}\PY{n}{e}\PY{p}{:}\PY{+w}{ }\PY{n+nc}{Expr}\PY{p}{,}\PY{+w}{ }\PY{n}{k}\PY{p}{:}\PY{+w}{ }\PY{n+nc}{Expr}\PY{p}{)}\PY{+w}{ }\PY{p}{\PYZhy{}\PYZgt{}}\PY{+w}{ }\PY{n+nc}{Expr}\PY{+w}{ }\PY{p}{\PYZob{}}
\PY{+w}{    }\PY{n}{e}\PY{p}{.}\PY{n}{pow}\PY{p}{(}\PY{n}{k}\PY{p}{)}
\PY{p}{\PYZcb{}}

\PY{k}{pub}\PY{+w}{ }\PY{k}{fn}\PY{+w}{ }\PY{n+nf}{U\PYZus{}inv}\PY{p}{(}\PY{n}{e}\PY{p}{:}\PY{+w}{ }\PY{n+nc}{Expr}\PY{p}{,}\PY{+w}{ }\PY{n}{k}\PY{p}{:}\PY{+w}{ }\PY{n+nc}{Expr}\PY{p}{)}\PY{+w}{ }\PY{p}{\PYZhy{}\PYZgt{}}\PY{+w}{ }\PY{n+nc}{Expr}\PY{+w}{ }\PY{p}{\PYZob{}}
\PY{+w}{    }\PY{n}{e}\PY{p}{.}\PY{n}{pow}\PY{p}{(}\PY{n}{lit}\PY{p}{(}\PY{l+m+mf}{1.0}\PY{p}{)}\PY{+w}{ }\PY{o}{/}\PY{+w}{ }\PY{n}{k}\PY{p}{)}
\PY{p}{\PYZcb{}}
\end{Verbatim}
\end{tcolorbox}

    Since our utility function has changed, Absolute Risk Aversion does as
well. As before, we can use simple calculus to determine it.

\[U(W_T, k) = W^k\] \[U'(W_T, k) = kW^{k-1}\]
\[U''(W_T, k) = k(k-1)W^{k-2}\]
\[A(W_T, k) = -\frac{k(k-1)W^{k-2}}{kW^{k-1}} = -\frac{k-1}{W}\]

The risk attitude of the investor is risk-averse when \(A\) is positive,
and risk-taking when \(A\) is negative. This is because the power
function is convex when \(k > 1\) e.g.~\(U(W, k) = W^2\) and concave
when \(k < 1\) e.g.~\(U(W, k) = W^{\tfrac{1}{2}}\). When \(k = 0\), the
utility function is linear (\(U(W, k) = W\)) and therefore risk-neutral.

\[\text{risk attitude} = \begin{cases} \text{risk-averse} & k < 1 \\ \text{risk-neutral} & k = 1 \\ \text{risk-taking} & k > 1 \end{cases}\]

    \begin{tcolorbox}[breakable, size=fbox, boxrule=1pt, pad at break*=1mm,colback=cellbackground, colframe=cellborder]
\prompt{In}{incolor}{26}{\boxspacing}
\begin{Verbatim}[commandchars=\\\{\}]
\PY{k+kd}{let}\PY{+w}{ }\PY{n}{A}\PY{+w}{ }\PY{o}{=}\PY{+w}{ }\PY{o}{\PYZhy{}}\PY{p}{(}\PY{n}{k}\PY{p}{.}\PY{n}{clone}\PY{p}{(}\PY{p}{)}\PY{+w}{ }\PY{o}{\PYZhy{}}\PY{+w}{ }\PY{n}{lit}\PY{p}{(}\PY{l+m+mf}{1.0}\PY{p}{)}\PY{p}{)}\PY{o}{/}\PY{+w}{ }\PY{n}{W\PYZus{}0}\PY{p}{.}\PY{n}{clone}\PY{p}{(}\PY{p}{)}\PY{p}{;}

\PY{k+kd}{let}\PY{+w}{ }\PY{n}{risk\PYZus{}attitude}\PY{+w}{ }\PY{o}{=}\PY{+w}{ }\PY{p}{(}\PY{n}{when}\PY{p}{(}\PY{n}{k}\PY{p}{.}\PY{n}{clone}\PY{p}{(}\PY{p}{)}\PY{p}{.}\PY{n}{lt}\PY{p}{(}\PY{l+m+mi}{1}\PY{p}{)}\PY{p}{)}
\PY{+w}{    }\PY{p}{.}\PY{n}{then}\PY{p}{(}\PY{n}{lit}\PY{p}{(}\PY{l+s}{\PYZdq{}}\PY{l+s}{risk\PYZhy{}averse}\PY{l+s}{\PYZdq{}}\PY{p}{)}\PY{p}{)}
\PY{+w}{    }\PY{p}{.}\PY{n}{otherwise}\PY{p}{(}
\PY{+w}{        }\PY{n}{when}\PY{p}{(}\PY{n}{k}\PY{p}{.}\PY{n}{clone}\PY{p}{(}\PY{p}{)}\PY{p}{.}\PY{n}{gt}\PY{p}{(}\PY{l+m+mi}{1}\PY{p}{)}\PY{p}{)}
\PY{+w}{            }\PY{p}{.}\PY{n}{then}\PY{p}{(}\PY{n}{lit}\PY{p}{(}\PY{l+s}{\PYZdq{}}\PY{l+s}{risk\PYZhy{}taking}\PY{l+s}{\PYZdq{}}\PY{p}{)}\PY{p}{)}
\PY{+w}{            }\PY{p}{.}\PY{n}{otherwise}\PY{p}{(}\PY{n}{lit}\PY{p}{(}\PY{l+s}{\PYZdq{}}\PY{l+s}{risk\PYZhy{}neutral}\PY{l+s}{\PYZdq{}}\PY{p}{)}\PY{p}{)}\PY{p}{,}
\PY{+w}{    }\PY{p}{)}\PY{p}{)}
\PY{p}{.}\PY{n}{alias}\PY{p}{(}\PY{l+s}{\PYZdq{}}\PY{l+s}{risk attitude}\PY{l+s}{\PYZdq{}}\PY{p}{)}\PY{p}{;}

\PY{n}{LaTeXDataFrame}\PY{p}{(}\PY{n}{df}\PY{p}{.}\PY{n}{clone}\PY{p}{(}\PY{p}{)}\PY{p}{.}\PY{n}{lazy}\PY{p}{(}\PY{p}{)}\PY{p}{.}\PY{n}{select}\PY{p}{(}\PY{p}{[}\PY{n}{k}\PY{p}{.}\PY{n}{clone}\PY{p}{(}\PY{p}{)}\PY{p}{,}\PY{+w}{ }\PY{n}{risk\PYZus{}attitude}\PY{p}{.}\PY{n}{clone}\PY{p}{(}\PY{p}{)}\PY{p}{]}\PY{p}{)}\PY{p}{.}\PY{n}{collect}\PY{p}{(}\PY{p}{)}\PY{o}{?}\PY{p}{)}\PY{p}{.}\PY{n}{evcxr\PYZus{}display}\PY{p}{(}\PY{p}{)}\PY{p}{;}
\end{Verbatim}
\end{tcolorbox}
 
            
\prompt{Out}{outcolor}{26}{}
    
    \begin{tabularx}{\textwidth}{|*2{p{\dimexpr\textwidth/2-2\tabcolsep\relax}|}}
\hlinek & risk attitude \\
\hline
0.5000 & risk-averse \\
\hline
\end{tabularx}

    

    We can redefine our computations with this new utility function.

    \begin{tcolorbox}[breakable, size=fbox, boxrule=1pt, pad at break*=1mm,colback=cellbackground, colframe=cellborder]
\prompt{In}{incolor}{27}{\boxspacing}
\begin{Verbatim}[commandchars=\\\{\}]
\PY{k+kd}{let}\PY{+w}{ }\PY{n}{EUW\PYZus{}T}\PY{+w}{ }\PY{o}{=}\PY{+w}{ }\PY{p}{(}\PY{n}{p\PYZus{}w}\PY{p}{.}\PY{n}{clone}\PY{p}{(}\PY{p}{)}
\PY{+w}{    }\PY{o}{*}\PY{+w}{ }\PY{n}{U}\PY{p}{(}\PY{n}{W\PYZus{}0}\PY{p}{.}\PY{n}{clone}\PY{p}{(}\PY{p}{)}\PY{+w}{ }\PY{o}{+}\PY{+w}{ }\PY{n}{w}\PY{p}{.}\PY{n}{clone}\PY{p}{(}\PY{p}{)}\PY{p}{,}\PY{+w}{ }\PY{n}{k}\PY{p}{.}\PY{n}{clone}\PY{p}{(}\PY{p}{)}\PY{p}{)}
\PY{+w}{    }\PY{o}{+}\PY{+w}{ }\PY{n}{p\PYZus{}l}\PY{p}{.}\PY{n}{clone}\PY{p}{(}\PY{p}{)}\PY{+w}{ }\PY{o}{*}\PY{+w}{ }\PY{n}{U}\PY{p}{(}\PY{n}{W\PYZus{}0}\PY{p}{.}\PY{n}{clone}\PY{p}{(}\PY{p}{)}\PY{+w}{ }\PY{o}{\PYZhy{}}\PY{+w}{ }\PY{n}{l}\PY{p}{.}\PY{n}{clone}\PY{p}{(}\PY{p}{)}\PY{p}{,}\PY{+w}{ }\PY{n}{k}\PY{p}{.}\PY{n}{clone}\PY{p}{(}\PY{p}{)}\PY{p}{)}\PY{p}{)}
\PY{+w}{    }\PY{p}{.}\PY{n}{alias}\PY{p}{(}\PY{l+s}{\PYZdq{}}\PY{l+s+se}{\PYZbs{}\PYZbs{}}\PY{l+s}{mathbb\PYZob{}E\PYZcb{}[U(W\PYZus{}T)]}\PY{l+s}{\PYZdq{}}\PY{p}{)}\PY{p}{;}

\PY{k+kd}{let}\PY{+w}{ }\PY{n}{CE}\PY{+w}{ }\PY{o}{=}\PY{+w}{ }\PY{p}{(}\PY{n}{U\PYZus{}inv}\PY{p}{(}\PY{n}{EUW\PYZus{}T}\PY{p}{.}\PY{n}{clone}\PY{p}{(}\PY{p}{)}\PY{p}{,}\PY{+w}{ }\PY{n}{k}\PY{p}{.}\PY{n}{clone}\PY{p}{(}\PY{p}{)}\PY{p}{)}\PY{p}{)}\PY{p}{.}\PY{n}{alias}\PY{p}{(}\PY{l+s}{\PYZdq{}}\PY{l+s}{CE}\PY{l+s}{\PYZdq{}}\PY{p}{)}\PY{p}{;}

\PY{k+kd}{let}\PY{+w}{ }\PY{n}{EW\PYZus{}T}\PY{+w}{ }\PY{o}{=}\PY{+w}{ }\PY{p}{(}\PY{n}{p\PYZus{}w}\PY{p}{.}\PY{n}{clone}\PY{p}{(}\PY{p}{)}
\PY{+w}{    }\PY{o}{*}\PY{+w}{ }\PY{p}{(}\PY{n}{W\PYZus{}0}\PY{p}{.}\PY{n}{clone}\PY{p}{(}\PY{p}{)}\PY{+w}{ }\PY{o}{+}\PY{+w}{ }\PY{n}{w}\PY{p}{.}\PY{n}{clone}\PY{p}{(}\PY{p}{)}\PY{p}{)}
\PY{+w}{    }\PY{o}{+}\PY{+w}{ }\PY{n}{p\PYZus{}l}\PY{p}{.}\PY{n}{clone}\PY{p}{(}\PY{p}{)}\PY{+w}{ }\PY{o}{*}\PY{+w}{ }\PY{p}{(}\PY{n}{W\PYZus{}0}\PY{p}{.}\PY{n}{clone}\PY{p}{(}\PY{p}{)}\PY{+w}{ }\PY{o}{\PYZhy{}}\PY{+w}{ }\PY{n}{l}\PY{p}{.}\PY{n}{clone}\PY{p}{(}\PY{p}{)}\PY{p}{)}\PY{p}{)}
\PY{+w}{    }\PY{p}{.}\PY{n}{alias}\PY{p}{(}\PY{l+s}{\PYZdq{}}\PY{l+s+se}{\PYZbs{}\PYZbs{}}\PY{l+s}{mathbb\PYZob{}E\PYZcb{}[W\PYZus{}T]}\PY{l+s}{\PYZdq{}}\PY{p}{)}\PY{p}{;}
\PY{k+kd}{let}\PY{+w}{ }\PY{n}{RP}\PY{+w}{ }\PY{o}{=}\PY{+w}{ }\PY{p}{(}\PY{n}{EW\PYZus{}T}\PY{p}{.}\PY{n}{clone}\PY{p}{(}\PY{p}{)}\PY{+w}{ }\PY{o}{\PYZhy{}}\PY{+w}{ }\PY{n}{CE}\PY{p}{.}\PY{n}{clone}\PY{p}{(}\PY{p}{)}\PY{p}{)}\PY{p}{.}\PY{n}{alias}\PY{p}{(}\PY{l+s}{\PYZdq{}}\PY{l+s}{RP}\PY{l+s}{\PYZdq{}}\PY{p}{)}\PY{p}{;}

\PY{k+kd}{let}\PY{+w}{ }\PY{n}{W\PYZus{}Tstar}\PY{+w}{ }\PY{o}{=}\PY{+w}{ }\PY{n}{EW\PYZus{}T}\PY{p}{.}\PY{n}{clone}\PY{p}{(}\PY{p}{)}\PY{p}{.}\PY{n}{alias}\PY{p}{(}\PY{l+s}{\PYZdq{}}\PY{l+s}{W\PYZca{}*\PYZus{}T}\PY{l+s}{\PYZdq{}}\PY{p}{)}\PY{p}{;}

\PY{k+kd}{let}\PY{+w}{ }\PY{n}{var}\PY{+w}{ }\PY{o}{=}\PY{+w}{ }\PY{p}{(}\PY{n}{p\PYZus{}w}\PY{p}{.}\PY{n}{clone}\PY{p}{(}\PY{p}{)}\PY{+w}{ }\PY{o}{*}\PY{+w}{ }\PY{n}{w}\PY{p}{.}\PY{n}{clone}\PY{p}{(}\PY{p}{)}\PY{p}{.}\PY{n}{pow}\PY{p}{(}\PY{l+m+mi}{2}\PY{p}{)}
\PY{o}{+}\PY{+w}{ }\PY{n}{p\PYZus{}l}\PY{p}{.}\PY{n}{clone}\PY{p}{(}\PY{p}{)}\PY{+w}{ }\PY{o}{*}\PY{+w}{ }\PY{n}{l}\PY{p}{.}\PY{n}{clone}\PY{p}{(}\PY{p}{)}\PY{p}{.}\PY{n}{pow}\PY{p}{(}\PY{l+m+mi}{2}\PY{p}{)}\PY{p}{)}\PY{p}{.}\PY{n}{alias}\PY{p}{(}\PY{l+s}{\PYZdq{}}\PY{l+s+se}{\PYZbs{}\PYZbs{}}\PY{l+s}{sigma\PYZus{}x\PYZca{}2}\PY{l+s}{\PYZdq{}}\PY{p}{)}\PY{p}{;}

\PY{k+kd}{let}\PY{+w}{ }\PY{n}{y}\PY{+w}{ }\PY{o}{=}\PY{+w}{ }\PY{p}{(}\PY{o}{\PYZhy{}}\PY{n}{var}\PY{p}{.}\PY{n}{clone}\PY{p}{(}\PY{p}{)}\PY{+w}{ }\PY{o}{/}\PY{+w}{ }\PY{n}{lit}\PY{p}{(}\PY{l+m+mf}{2.0}\PY{p}{)}\PY{+w}{ }\PY{o}{*}\PY{+w}{ }\PY{n}{A}\PY{p}{.}\PY{n}{clone}\PY{p}{(}\PY{p}{)}\PY{p}{)}\PY{p}{.}\PY{n}{alias}\PY{p}{(}\PY{l+s}{\PYZdq{}}\PY{l+s}{y}\PY{l+s}{\PYZdq{}}\PY{p}{)}\PY{p}{;}

\PY{k+kd}{let}\PY{+w}{ }\PY{n}{approximation\PYZus{}error}\PY{+w}{ }\PY{o}{=}\PY{+w}{ }\PY{p}{(}\PY{n}{RP}\PY{p}{.}\PY{n}{clone}\PY{p}{(}\PY{p}{)}\PY{+w}{ }\PY{o}{\PYZhy{}}\PY{+w}{ }\PY{n}{y}\PY{p}{.}\PY{n}{clone}\PY{p}{(}\PY{p}{)}\PY{p}{)}\PY{p}{.}\PY{n}{alias}\PY{p}{(}\PY{l+s}{\PYZdq{}}\PY{l+s}{approximation error}\PY{l+s}{\PYZdq{}}\PY{p}{)}\PY{p}{;}

\PY{n}{LaTeXDataFrame}\PY{p}{(}\PY{n}{df}\PY{p}{.}\PY{n}{clone}\PY{p}{(}\PY{p}{)}\PY{p}{.}\PY{n}{lazy}\PY{p}{(}\PY{p}{)}\PY{p}{.}\PY{n}{select}\PY{p}{(}\PY{p}{[}\PY{n}{CE}\PY{p}{.}\PY{n}{clone}\PY{p}{(}\PY{p}{)}\PY{p}{,}\PY{+w}{ }\PY{n}{RP}\PY{p}{.}\PY{n}{clone}\PY{p}{(}\PY{p}{)}\PY{p}{,}\PY{+w}{ }\PY{n}{approximation\PYZus{}error}\PY{p}{.}\PY{n}{clone}\PY{p}{(}\PY{p}{)}\PY{p}{]}\PY{p}{)}\PY{p}{.}\PY{n}{collect}\PY{p}{(}\PY{p}{)}\PY{o}{?}\PY{p}{)}\PY{p}{.}\PY{n}{evcxr\PYZus{}display}\PY{p}{(}\PY{p}{)}\PY{p}{;}
\end{Verbatim}
\end{tcolorbox}
 
            
\prompt{Out}{outcolor}{27}{}
    
    \begin{tabularx}{\textwidth}{|*3{p{\dimexpr\textwidth/3-2\tabcolsep\relax}|}}
\hlineCE & RP & approximation error \\
\hline
980.1307 & 23.2026 & 43.5401 \\
\hline
\end{tabularx}

    

    Then recalculate with \(k = 2\).

    \begin{tcolorbox}[breakable, size=fbox, boxrule=1pt, pad at break*=1mm,colback=cellbackground, colframe=cellborder]
\prompt{In}{incolor}{28}{\boxspacing}
\begin{Verbatim}[commandchars=\\\{\}]
\PY{n}{df}\PY{p}{.}\PY{n}{with\PYZus{}column}\PY{p}{(}\PY{n}{Column}\PY{p}{::}\PY{n}{new}\PY{p}{(}
\PY{+w}{    }\PY{l+s}{\PYZdq{}}\PY{l+s}{k}\PY{l+s}{\PYZdq{}}\PY{p}{.}\PY{n}{into}\PY{p}{(}\PY{p}{)}\PY{p}{,}
\PY{+w}{    }\PY{p}{[}\PY{l+m+mf}{2.0}\PY{p}{]}\PY{p}{,}
\PY{p}{)}\PY{p}{)}\PY{o}{?}\PY{p}{;}

\PY{n}{LaTeXDataFrame}\PY{p}{(}\PY{n}{df}\PY{p}{.}\PY{n}{clone}\PY{p}{(}\PY{p}{)}\PY{p}{.}\PY{n}{lazy}\PY{p}{(}\PY{p}{)}\PY{p}{.}\PY{n}{select}\PY{p}{(}\PY{p}{[}\PY{n}{CE}\PY{p}{.}\PY{n}{clone}\PY{p}{(}\PY{p}{)}\PY{p}{,}\PY{+w}{ }\PY{n}{RP}\PY{p}{.}\PY{n}{clone}\PY{p}{(}\PY{p}{)}\PY{p}{,}\PY{+w}{ }\PY{n}{approximation\PYZus{}error}\PY{p}{.}\PY{n}{clone}\PY{p}{(}\PY{p}{)}\PY{p}{]}\PY{p}{)}\PY{p}{.}\PY{n}{collect}\PY{p}{(}\PY{p}{)}\PY{o}{?}\PY{p}{)}\PY{p}{.}\PY{n}{evcxr\PYZus{}display}\PY{p}{(}\PY{p}{)}\PY{p}{;}
\end{Verbatim}
\end{tcolorbox}
 
            
\prompt{Out}{outcolor}{28}{}
    
    \begin{tabularx}{\textwidth}{|*3{p{\dimexpr\textwidth/3-2\tabcolsep\relax}|}}
\hlineCE & RP & approximation error \\
\hline
1043.0804 & -39.7470 & -80.4220 \\
\hline
\end{tabularx}

    

    \subsection{3. Scaled Log Utility}\label{scaled-log-utility}

    We'll initialize some dummy data to test with.

    We are given mean returns \(\mu\) for each security, so we can simply
calculate \[\mathbb{E}[r_p(w)] = \mathbb{E}[w^T r] = w^T \mu\]

Following the general variance formula given covariance matrix
\(\Sigma\),

\[\sigma^2[r_p(w)] = \mathbb{V}[w^T r] = w^T \Sigma w\]

    The utility function we are using is \[U(r) = \ln(1 + \lambda r)\]

    The Taylor series expansion of this is given by the following:
\[U(0) = \ln(1 + \lambda (0)) = 0\]
\[U'(r) = \frac{\lambda}{1 + \lambda r}\]
\[U'(0) = \frac{\lambda}{1 + \lambda (0)} = \lambda\]
\[U''(r) = -\frac{\lambda^2}{(1 + \lambda r)^2}\]
\[U''(0) = -\frac{\lambda^2}{(1 + \lambda (0))^2} = -\lambda^2\]
\[U(r) \approx U(0) + U'(0) r + \frac{U''(0)}{2}r^2 = \lambda r - \frac{\lambda^2}{2} r^2\]

    In order to prove the approximation, we will take the expectation:
\[\mathbb{E}[U(r)] \approx \lambda \mathbb{E}[r] - \frac{\lambda^2}{2} \mathbb{E}[r^2]\]
We can take advantage of the definition of variance here.
\[\sigma^2(r) = \mathbb{E}[r^2] - \mathbb{E}^2[r]\] Note that because
\(\mu\) is small, \(\mathbb{E}^2[r]\) is small enough to erase from our
calculations, leading to \[\sigma^2(r) \approx \mathbb{E}[r^2]\] We can
resubstitute this back into our equation
\[\mathbb{E}[U(r)] \approx \lambda \mathbb{E}[r] - \frac{\lambda^2}{2} \sigma^2(r)\]
And divide by the constant \(\lambda\) as that does not impact
maximization.
\[\mathbb{E}[U(r)] \approx \mathbb{E}[r] - \frac{\lambda}{2} \sigma^2(r)\]
Thus the approximation is proven.


    % Add a bibliography block to the postdoc
    
    
    
\end{document}
