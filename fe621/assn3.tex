\documentclass[11pt]{article}

    \usepackage[breakable]{tcolorbox}
    \usepackage{parskip} % Stop auto-indenting (to mimic markdown behaviour)
    

    % Basic figure setup, for now with no caption control since it's done
    % automatically by Pandoc (which extracts ![](path) syntax from Markdown).
    \usepackage{graphicx}
    % Keep aspect ratio if custom image width or height is specified
    \setkeys{Gin}{keepaspectratio}
    % Maintain compatibility with old templates. Remove in nbconvert 6.0
    \let\Oldincludegraphics\includegraphics
    % Ensure that by default, figures have no caption (until we provide a
    % proper Figure object with a Caption API and a way to capture that
    % in the conversion process - todo).
    \usepackage{caption}
    \DeclareCaptionFormat{nocaption}{}
    \captionsetup{format=nocaption,aboveskip=0pt,belowskip=0pt}

    \usepackage{float}
    \floatplacement{figure}{H} % forces figures to be placed at the correct location
    \usepackage{xcolor} % Allow colors to be defined
    \usepackage{enumerate} % Needed for markdown enumerations to work
    \usepackage{geometry} % Used to adjust the document margins
    \usepackage{amsmath} % Equations
    \usepackage{amssymb} % Equations
    \usepackage{textcomp} % defines textquotesingle
    % Hack from http://tex.stackexchange.com/a/47451/13684:
    \AtBeginDocument{%
        \def\PYZsq{\textquotesingle}% Upright quotes in Pygmentized code
    }
    \usepackage{upquote} % Upright quotes for verbatim code
    \usepackage{eurosym} % defines \euro

    \usepackage{iftex}
    \ifPDFTeX
        \usepackage[T1]{fontenc}
        \IfFileExists{alphabeta.sty}{
              \usepackage{alphabeta}
          }{
              \usepackage[mathletters]{ucs}
              \usepackage[utf8x]{inputenc}
          }
    \else
        \usepackage{fontspec}
        \usepackage{unicode-math}
    \fi

    \usepackage{fancyvrb} % verbatim replacement that allows latex
    \usepackage{grffile} % extends the file name processing of package graphics
                         % to support a larger range
    \makeatletter % fix for old versions of grffile with XeLaTeX
    \@ifpackagelater{grffile}{2019/11/01}
    {
      % Do nothing on new versions
    }
    {
      \def\Gread@@xetex#1{%
        \IfFileExists{"\Gin@base".bb}%
        {\Gread@eps{\Gin@base.bb}}%
        {\Gread@@xetex@aux#1}%
      }
    }
    \makeatother
    \usepackage[Export]{adjustbox} % Used to constrain images to a maximum size
    \adjustboxset{max size={0.9\linewidth}{0.9\paperheight}}

    % The hyperref package gives us a pdf with properly built
    % internal navigation ('pdf bookmarks' for the table of contents,
    % internal cross-reference links, web links for URLs, etc.)
    \usepackage{hyperref}
    % The default LaTeX title has an obnoxious amount of whitespace. By default,
    % titling removes some of it. It also provides customization options.
    \usepackage{titling}
    \usepackage{longtable} % longtable support required by pandoc >1.10
    \usepackage{booktabs}  % table support for pandoc > 1.12.2
    \usepackage{array}     % table support for pandoc >= 2.11.3
    \usepackage{calc}      % table minipage width calculation for pandoc >= 2.11.1
    \usepackage[inline]{enumitem} % IRkernel/repr support (it uses the enumerate* environment)
    \usepackage[normalem]{ulem} % ulem is needed to support strikethroughs (\sout)
                                % normalem makes italics be italics, not underlines
    \usepackage{soul}      % strikethrough (\st) support for pandoc >= 3.0.0
    \usepackage{mathrsfs}
    

    
    % Colors for the hyperref package
    \definecolor{urlcolor}{rgb}{0,.145,.698}
    \definecolor{linkcolor}{rgb}{.71,0.21,0.01}
    \definecolor{citecolor}{rgb}{.12,.54,.11}

    % ANSI colors
    \definecolor{ansi-black}{HTML}{3E424D}
    \definecolor{ansi-black-intense}{HTML}{282C36}
    \definecolor{ansi-red}{HTML}{E75C58}
    \definecolor{ansi-red-intense}{HTML}{B22B31}
    \definecolor{ansi-green}{HTML}{00A250}
    \definecolor{ansi-green-intense}{HTML}{007427}
    \definecolor{ansi-yellow}{HTML}{DDB62B}
    \definecolor{ansi-yellow-intense}{HTML}{B27D12}
    \definecolor{ansi-blue}{HTML}{208FFB}
    \definecolor{ansi-blue-intense}{HTML}{0065CA}
    \definecolor{ansi-magenta}{HTML}{D160C4}
    \definecolor{ansi-magenta-intense}{HTML}{A03196}
    \definecolor{ansi-cyan}{HTML}{60C6C8}
    \definecolor{ansi-cyan-intense}{HTML}{258F8F}
    \definecolor{ansi-white}{HTML}{C5C1B4}
    \definecolor{ansi-white-intense}{HTML}{A1A6B2}
    \definecolor{ansi-default-inverse-fg}{HTML}{FFFFFF}
    \definecolor{ansi-default-inverse-bg}{HTML}{000000}

    % common color for the border for error outputs.
    \definecolor{outerrorbackground}{HTML}{FFDFDF}

    % commands and environments needed by pandoc snippets
    % extracted from the output of `pandoc -s`
    \providecommand{\tightlist}{%
      \setlength{\itemsep}{0pt}\setlength{\parskip}{0pt}}
    \DefineVerbatimEnvironment{Highlighting}{Verbatim}{commandchars=\\\{\}}
    % Add ',fontsize=\small' for more characters per line
    \newenvironment{Shaded}{}{}
    \newcommand{\KeywordTok}[1]{\textcolor[rgb]{0.00,0.44,0.13}{\textbf{{#1}}}}
    \newcommand{\DataTypeTok}[1]{\textcolor[rgb]{0.56,0.13,0.00}{{#1}}}
    \newcommand{\DecValTok}[1]{\textcolor[rgb]{0.25,0.63,0.44}{{#1}}}
    \newcommand{\BaseNTok}[1]{\textcolor[rgb]{0.25,0.63,0.44}{{#1}}}
    \newcommand{\FloatTok}[1]{\textcolor[rgb]{0.25,0.63,0.44}{{#1}}}
    \newcommand{\CharTok}[1]{\textcolor[rgb]{0.25,0.44,0.63}{{#1}}}
    \newcommand{\StringTok}[1]{\textcolor[rgb]{0.25,0.44,0.63}{{#1}}}
    \newcommand{\CommentTok}[1]{\textcolor[rgb]{0.38,0.63,0.69}{\textit{{#1}}}}
    \newcommand{\OtherTok}[1]{\textcolor[rgb]{0.00,0.44,0.13}{{#1}}}
    \newcommand{\AlertTok}[1]{\textcolor[rgb]{1.00,0.00,0.00}{\textbf{{#1}}}}
    \newcommand{\FunctionTok}[1]{\textcolor[rgb]{0.02,0.16,0.49}{{#1}}}
    \newcommand{\RegionMarkerTok}[1]{{#1}}
    \newcommand{\ErrorTok}[1]{\textcolor[rgb]{1.00,0.00,0.00}{\textbf{{#1}}}}
    \newcommand{\NormalTok}[1]{{#1}}

    % Additional commands for more recent versions of Pandoc
    \newcommand{\ConstantTok}[1]{\textcolor[rgb]{0.53,0.00,0.00}{{#1}}}
    \newcommand{\SpecialCharTok}[1]{\textcolor[rgb]{0.25,0.44,0.63}{{#1}}}
    \newcommand{\VerbatimStringTok}[1]{\textcolor[rgb]{0.25,0.44,0.63}{{#1}}}
    \newcommand{\SpecialStringTok}[1]{\textcolor[rgb]{0.73,0.40,0.53}{{#1}}}
    \newcommand{\ImportTok}[1]{{#1}}
    \newcommand{\DocumentationTok}[1]{\textcolor[rgb]{0.73,0.13,0.13}{\textit{{#1}}}}
    \newcommand{\AnnotationTok}[1]{\textcolor[rgb]{0.38,0.63,0.69}{\textbf{\textit{{#1}}}}}
    \newcommand{\CommentVarTok}[1]{\textcolor[rgb]{0.38,0.63,0.69}{\textbf{\textit{{#1}}}}}
    \newcommand{\VariableTok}[1]{\textcolor[rgb]{0.10,0.09,0.49}{{#1}}}
    \newcommand{\ControlFlowTok}[1]{\textcolor[rgb]{0.00,0.44,0.13}{\textbf{{#1}}}}
    \newcommand{\OperatorTok}[1]{\textcolor[rgb]{0.40,0.40,0.40}{{#1}}}
    \newcommand{\BuiltInTok}[1]{{#1}}
    \newcommand{\ExtensionTok}[1]{{#1}}
    \newcommand{\PreprocessorTok}[1]{\textcolor[rgb]{0.74,0.48,0.00}{{#1}}}
    \newcommand{\AttributeTok}[1]{\textcolor[rgb]{0.49,0.56,0.16}{{#1}}}
    \newcommand{\InformationTok}[1]{\textcolor[rgb]{0.38,0.63,0.69}{\textbf{\textit{{#1}}}}}
    \newcommand{\WarningTok}[1]{\textcolor[rgb]{0.38,0.63,0.69}{\textbf{\textit{{#1}}}}}
    \makeatletter
    \newsavebox\pandoc@box
    \newcommand*\pandocbounded[1]{%
      \sbox\pandoc@box{#1}%
      % scaling factors for width and height
      \Gscale@div\@tempa\textheight{\dimexpr\ht\pandoc@box+\dp\pandoc@box\relax}%
      \Gscale@div\@tempb\linewidth{\wd\pandoc@box}%
      % select the smaller of both
      \ifdim\@tempb\p@<\@tempa\p@
        \let\@tempa\@tempb
      \fi
      % scaling accordingly (\@tempa < 1)
      \ifdim\@tempa\p@<\p@
        \scalebox{\@tempa}{\usebox\pandoc@box}%
      % scaling not needed, use as it is
      \else
        \usebox{\pandoc@box}%
      \fi
    }
    \makeatother

    % Define a nice break command that doesn't care if a line doesn't already
    % exist.
    \def\br{\hspace*{\fill} \\* }
    % Math Jax compatibility definitions
    \def\gt{>}
    \def\lt{<}
    \let\Oldtex\TeX
    \let\Oldlatex\LaTeX
    \renewcommand{\TeX}{\textrm{\Oldtex}}
    \renewcommand{\LaTeX}{\textrm{\Oldlatex}}
    % Document parameters
    % Document title
    \title{assn3}
    
    
    
    
    
    
    
% Pygments definitions
\makeatletter
\def\PY@reset{\let\PY@it=\relax \let\PY@bf=\relax%
    \let\PY@ul=\relax \let\PY@tc=\relax%
    \let\PY@bc=\relax \let\PY@ff=\relax}
\def\PY@tok#1{\csname PY@tok@#1\endcsname}
\def\PY@toks#1+{\ifx\relax#1\empty\else%
    \PY@tok{#1}\expandafter\PY@toks\fi}
\def\PY@do#1{\PY@bc{\PY@tc{\PY@ul{%
    \PY@it{\PY@bf{\PY@ff{#1}}}}}}}
\def\PY#1#2{\PY@reset\PY@toks#1+\relax+\PY@do{#2}}

\@namedef{PY@tok@w}{\def\PY@tc##1{\textcolor[rgb]{0.73,0.73,0.73}{##1}}}
\@namedef{PY@tok@c}{\let\PY@it=\textit\def\PY@tc##1{\textcolor[rgb]{0.24,0.48,0.48}{##1}}}
\@namedef{PY@tok@cp}{\def\PY@tc##1{\textcolor[rgb]{0.61,0.40,0.00}{##1}}}
\@namedef{PY@tok@k}{\let\PY@bf=\textbf\def\PY@tc##1{\textcolor[rgb]{0.00,0.50,0.00}{##1}}}
\@namedef{PY@tok@kp}{\def\PY@tc##1{\textcolor[rgb]{0.00,0.50,0.00}{##1}}}
\@namedef{PY@tok@kt}{\def\PY@tc##1{\textcolor[rgb]{0.69,0.00,0.25}{##1}}}
\@namedef{PY@tok@o}{\def\PY@tc##1{\textcolor[rgb]{0.40,0.40,0.40}{##1}}}
\@namedef{PY@tok@ow}{\let\PY@bf=\textbf\def\PY@tc##1{\textcolor[rgb]{0.67,0.13,1.00}{##1}}}
\@namedef{PY@tok@nb}{\def\PY@tc##1{\textcolor[rgb]{0.00,0.50,0.00}{##1}}}
\@namedef{PY@tok@nf}{\def\PY@tc##1{\textcolor[rgb]{0.00,0.00,1.00}{##1}}}
\@namedef{PY@tok@nc}{\let\PY@bf=\textbf\def\PY@tc##1{\textcolor[rgb]{0.00,0.00,1.00}{##1}}}
\@namedef{PY@tok@nn}{\let\PY@bf=\textbf\def\PY@tc##1{\textcolor[rgb]{0.00,0.00,1.00}{##1}}}
\@namedef{PY@tok@ne}{\let\PY@bf=\textbf\def\PY@tc##1{\textcolor[rgb]{0.80,0.25,0.22}{##1}}}
\@namedef{PY@tok@nv}{\def\PY@tc##1{\textcolor[rgb]{0.10,0.09,0.49}{##1}}}
\@namedef{PY@tok@no}{\def\PY@tc##1{\textcolor[rgb]{0.53,0.00,0.00}{##1}}}
\@namedef{PY@tok@nl}{\def\PY@tc##1{\textcolor[rgb]{0.46,0.46,0.00}{##1}}}
\@namedef{PY@tok@ni}{\let\PY@bf=\textbf\def\PY@tc##1{\textcolor[rgb]{0.44,0.44,0.44}{##1}}}
\@namedef{PY@tok@na}{\def\PY@tc##1{\textcolor[rgb]{0.41,0.47,0.13}{##1}}}
\@namedef{PY@tok@nt}{\let\PY@bf=\textbf\def\PY@tc##1{\textcolor[rgb]{0.00,0.50,0.00}{##1}}}
\@namedef{PY@tok@nd}{\def\PY@tc##1{\textcolor[rgb]{0.67,0.13,1.00}{##1}}}
\@namedef{PY@tok@s}{\def\PY@tc##1{\textcolor[rgb]{0.73,0.13,0.13}{##1}}}
\@namedef{PY@tok@sd}{\let\PY@it=\textit\def\PY@tc##1{\textcolor[rgb]{0.73,0.13,0.13}{##1}}}
\@namedef{PY@tok@si}{\let\PY@bf=\textbf\def\PY@tc##1{\textcolor[rgb]{0.64,0.35,0.47}{##1}}}
\@namedef{PY@tok@se}{\let\PY@bf=\textbf\def\PY@tc##1{\textcolor[rgb]{0.67,0.36,0.12}{##1}}}
\@namedef{PY@tok@sr}{\def\PY@tc##1{\textcolor[rgb]{0.64,0.35,0.47}{##1}}}
\@namedef{PY@tok@ss}{\def\PY@tc##1{\textcolor[rgb]{0.10,0.09,0.49}{##1}}}
\@namedef{PY@tok@sx}{\def\PY@tc##1{\textcolor[rgb]{0.00,0.50,0.00}{##1}}}
\@namedef{PY@tok@m}{\def\PY@tc##1{\textcolor[rgb]{0.40,0.40,0.40}{##1}}}
\@namedef{PY@tok@gh}{\let\PY@bf=\textbf\def\PY@tc##1{\textcolor[rgb]{0.00,0.00,0.50}{##1}}}
\@namedef{PY@tok@gu}{\let\PY@bf=\textbf\def\PY@tc##1{\textcolor[rgb]{0.50,0.00,0.50}{##1}}}
\@namedef{PY@tok@gd}{\def\PY@tc##1{\textcolor[rgb]{0.63,0.00,0.00}{##1}}}
\@namedef{PY@tok@gi}{\def\PY@tc##1{\textcolor[rgb]{0.00,0.52,0.00}{##1}}}
\@namedef{PY@tok@gr}{\def\PY@tc##1{\textcolor[rgb]{0.89,0.00,0.00}{##1}}}
\@namedef{PY@tok@ge}{\let\PY@it=\textit}
\@namedef{PY@tok@gs}{\let\PY@bf=\textbf}
\@namedef{PY@tok@ges}{\let\PY@bf=\textbf\let\PY@it=\textit}
\@namedef{PY@tok@gp}{\let\PY@bf=\textbf\def\PY@tc##1{\textcolor[rgb]{0.00,0.00,0.50}{##1}}}
\@namedef{PY@tok@go}{\def\PY@tc##1{\textcolor[rgb]{0.44,0.44,0.44}{##1}}}
\@namedef{PY@tok@gt}{\def\PY@tc##1{\textcolor[rgb]{0.00,0.27,0.87}{##1}}}
\@namedef{PY@tok@err}{\def\PY@bc##1{{\setlength{\fboxsep}{\string -\fboxrule}\fcolorbox[rgb]{1.00,0.00,0.00}{1,1,1}{\strut ##1}}}}
\@namedef{PY@tok@kc}{\let\PY@bf=\textbf\def\PY@tc##1{\textcolor[rgb]{0.00,0.50,0.00}{##1}}}
\@namedef{PY@tok@kd}{\let\PY@bf=\textbf\def\PY@tc##1{\textcolor[rgb]{0.00,0.50,0.00}{##1}}}
\@namedef{PY@tok@kn}{\let\PY@bf=\textbf\def\PY@tc##1{\textcolor[rgb]{0.00,0.50,0.00}{##1}}}
\@namedef{PY@tok@kr}{\let\PY@bf=\textbf\def\PY@tc##1{\textcolor[rgb]{0.00,0.50,0.00}{##1}}}
\@namedef{PY@tok@bp}{\def\PY@tc##1{\textcolor[rgb]{0.00,0.50,0.00}{##1}}}
\@namedef{PY@tok@fm}{\def\PY@tc##1{\textcolor[rgb]{0.00,0.00,1.00}{##1}}}
\@namedef{PY@tok@vc}{\def\PY@tc##1{\textcolor[rgb]{0.10,0.09,0.49}{##1}}}
\@namedef{PY@tok@vg}{\def\PY@tc##1{\textcolor[rgb]{0.10,0.09,0.49}{##1}}}
\@namedef{PY@tok@vi}{\def\PY@tc##1{\textcolor[rgb]{0.10,0.09,0.49}{##1}}}
\@namedef{PY@tok@vm}{\def\PY@tc##1{\textcolor[rgb]{0.10,0.09,0.49}{##1}}}
\@namedef{PY@tok@sa}{\def\PY@tc##1{\textcolor[rgb]{0.73,0.13,0.13}{##1}}}
\@namedef{PY@tok@sb}{\def\PY@tc##1{\textcolor[rgb]{0.73,0.13,0.13}{##1}}}
\@namedef{PY@tok@sc}{\def\PY@tc##1{\textcolor[rgb]{0.73,0.13,0.13}{##1}}}
\@namedef{PY@tok@dl}{\def\PY@tc##1{\textcolor[rgb]{0.73,0.13,0.13}{##1}}}
\@namedef{PY@tok@s2}{\def\PY@tc##1{\textcolor[rgb]{0.73,0.13,0.13}{##1}}}
\@namedef{PY@tok@sh}{\def\PY@tc##1{\textcolor[rgb]{0.73,0.13,0.13}{##1}}}
\@namedef{PY@tok@s1}{\def\PY@tc##1{\textcolor[rgb]{0.73,0.13,0.13}{##1}}}
\@namedef{PY@tok@mb}{\def\PY@tc##1{\textcolor[rgb]{0.40,0.40,0.40}{##1}}}
\@namedef{PY@tok@mf}{\def\PY@tc##1{\textcolor[rgb]{0.40,0.40,0.40}{##1}}}
\@namedef{PY@tok@mh}{\def\PY@tc##1{\textcolor[rgb]{0.40,0.40,0.40}{##1}}}
\@namedef{PY@tok@mi}{\def\PY@tc##1{\textcolor[rgb]{0.40,0.40,0.40}{##1}}}
\@namedef{PY@tok@il}{\def\PY@tc##1{\textcolor[rgb]{0.40,0.40,0.40}{##1}}}
\@namedef{PY@tok@mo}{\def\PY@tc##1{\textcolor[rgb]{0.40,0.40,0.40}{##1}}}
\@namedef{PY@tok@ch}{\let\PY@it=\textit\def\PY@tc##1{\textcolor[rgb]{0.24,0.48,0.48}{##1}}}
\@namedef{PY@tok@cm}{\let\PY@it=\textit\def\PY@tc##1{\textcolor[rgb]{0.24,0.48,0.48}{##1}}}
\@namedef{PY@tok@cpf}{\let\PY@it=\textit\def\PY@tc##1{\textcolor[rgb]{0.24,0.48,0.48}{##1}}}
\@namedef{PY@tok@c1}{\let\PY@it=\textit\def\PY@tc##1{\textcolor[rgb]{0.24,0.48,0.48}{##1}}}
\@namedef{PY@tok@cs}{\let\PY@it=\textit\def\PY@tc##1{\textcolor[rgb]{0.24,0.48,0.48}{##1}}}

\def\PYZbs{\char`\\}
\def\PYZus{\char`\_}
\def\PYZob{\char`\{}
\def\PYZcb{\char`\}}
\def\PYZca{\char`\^}
\def\PYZam{\char`\&}
\def\PYZlt{\char`\<}
\def\PYZgt{\char`\>}
\def\PYZsh{\char`\#}
\def\PYZpc{\char`\%}
\def\PYZdl{\char`\$}
\def\PYZhy{\char`\-}
\def\PYZsq{\char`\'}
\def\PYZdq{\char`\"}
\def\PYZti{\char`\~}
% for compatibility with earlier versions
\def\PYZat{@}
\def\PYZlb{[}
\def\PYZrb{]}
\makeatother


    % For linebreaks inside Verbatim environment from package fancyvrb.
    \makeatletter
        \newbox\Wrappedcontinuationbox
        \newbox\Wrappedvisiblespacebox
        \newcommand*\Wrappedvisiblespace {\textcolor{red}{\textvisiblespace}}
        \newcommand*\Wrappedcontinuationsymbol {\textcolor{red}{\llap{\tiny$\m@th\hookrightarrow$}}}
        \newcommand*\Wrappedcontinuationindent {3ex }
        \newcommand*\Wrappedafterbreak {\kern\Wrappedcontinuationindent\copy\Wrappedcontinuationbox}
        % Take advantage of the already applied Pygments mark-up to insert
        % potential linebreaks for TeX processing.
        %        {, <, #, %, $, ' and ": go to next line.
        %        _, }, ^, &, >, - and ~: stay at end of broken line.
        % Use of \textquotesingle for straight quote.
        \newcommand*\Wrappedbreaksatspecials {%
            \def\PYGZus{\discretionary{\char`\_}{\Wrappedafterbreak}{\char`\_}}%
            \def\PYGZob{\discretionary{}{\Wrappedafterbreak\char`\{}{\char`\{}}%
            \def\PYGZcb{\discretionary{\char`\}}{\Wrappedafterbreak}{\char`\}}}%
            \def\PYGZca{\discretionary{\char`\^}{\Wrappedafterbreak}{\char`\^}}%
            \def\PYGZam{\discretionary{\char`\&}{\Wrappedafterbreak}{\char`\&}}%
            \def\PYGZlt{\discretionary{}{\Wrappedafterbreak\char`\<}{\char`\<}}%
            \def\PYGZgt{\discretionary{\char`\>}{\Wrappedafterbreak}{\char`\>}}%
            \def\PYGZsh{\discretionary{}{\Wrappedafterbreak\char`\#}{\char`\#}}%
            \def\PYGZpc{\discretionary{}{\Wrappedafterbreak\char`\%}{\char`\%}}%
            \def\PYGZdl{\discretionary{}{\Wrappedafterbreak\char`\$}{\char`\$}}%
            \def\PYGZhy{\discretionary{\char`\-}{\Wrappedafterbreak}{\char`\-}}%
            \def\PYGZsq{\discretionary{}{\Wrappedafterbreak\textquotesingle}{\textquotesingle}}%
            \def\PYGZdq{\discretionary{}{\Wrappedafterbreak\char`\"}{\char`\"}}%
            \def\PYGZti{\discretionary{\char`\~}{\Wrappedafterbreak}{\char`\~}}%
        }
        % Some characters . , ; ? ! / are not pygmentized.
        % This macro makes them "active" and they will insert potential linebreaks
        \newcommand*\Wrappedbreaksatpunct {%
            \lccode`\~`\.\lowercase{\def~}{\discretionary{\hbox{\char`\.}}{\Wrappedafterbreak}{\hbox{\char`\.}}}%
            \lccode`\~`\,\lowercase{\def~}{\discretionary{\hbox{\char`\,}}{\Wrappedafterbreak}{\hbox{\char`\,}}}%
            \lccode`\~`\;\lowercase{\def~}{\discretionary{\hbox{\char`\;}}{\Wrappedafterbreak}{\hbox{\char`\;}}}%
            \lccode`\~`\:\lowercase{\def~}{\discretionary{\hbox{\char`\:}}{\Wrappedafterbreak}{\hbox{\char`\:}}}%
            \lccode`\~`\?\lowercase{\def~}{\discretionary{\hbox{\char`\?}}{\Wrappedafterbreak}{\hbox{\char`\?}}}%
            \lccode`\~`\!\lowercase{\def~}{\discretionary{\hbox{\char`\!}}{\Wrappedafterbreak}{\hbox{\char`\!}}}%
            \lccode`\~`\/\lowercase{\def~}{\discretionary{\hbox{\char`\/}}{\Wrappedafterbreak}{\hbox{\char`\/}}}%
            \catcode`\.\active
            \catcode`\,\active
            \catcode`\;\active
            \catcode`\:\active
            \catcode`\?\active
            \catcode`\!\active
            \catcode`\/\active
            \lccode`\~`\~
        }
    \makeatother

    \let\OriginalVerbatim=\Verbatim
    \makeatletter
    \renewcommand{\Verbatim}[1][1]{%
        %\parskip\z@skip
        \sbox\Wrappedcontinuationbox {\Wrappedcontinuationsymbol}%
        \sbox\Wrappedvisiblespacebox {\FV@SetupFont\Wrappedvisiblespace}%
        \def\FancyVerbFormatLine ##1{\hsize\linewidth
            \vtop{\raggedright\hyphenpenalty\z@\exhyphenpenalty\z@
                \doublehyphendemerits\z@\finalhyphendemerits\z@
                \strut ##1\strut}%
        }%
        % If the linebreak is at a space, the latter will be displayed as visible
        % space at end of first line, and a continuation symbol starts next line.
        % Stretch/shrink are however usually zero for typewriter font.
        \def\FV@Space {%
            \nobreak\hskip\z@ plus\fontdimen3\font minus\fontdimen4\font
            \discretionary{\copy\Wrappedvisiblespacebox}{\Wrappedafterbreak}
            {\kern\fontdimen2\font}%
        }%

        % Allow breaks at special characters using \PYG... macros.
        \Wrappedbreaksatspecials
        % Breaks at punctuation characters . , ; ? ! and / need catcode=\active
        \OriginalVerbatim[#1,codes*=\Wrappedbreaksatpunct]%
    }
    \makeatother

    % Exact colors from NB
    \definecolor{incolor}{HTML}{303F9F}
    \definecolor{outcolor}{HTML}{D84315}
    \definecolor{cellborder}{HTML}{CFCFCF}
    \definecolor{cellbackground}{HTML}{F7F7F7}

    % prompt
    \makeatletter
    \newcommand{\boxspacing}{\kern\kvtcb@left@rule\kern\kvtcb@boxsep}
    \makeatother
    \newcommand{\prompt}[4]{
        {\ttfamily\llap{{\color{#2}[#3]:\hspace{3pt}#4}}\vspace{-\baselineskip}}
    }
    

    
    % Prevent overflowing lines due to hard-to-break entities
    \sloppy
    % Setup hyperref package
    \hypersetup{
      breaklinks=true,  % so long urls are correctly broken across lines
      colorlinks=true,
      urlcolor=urlcolor,
      linkcolor=linkcolor,
      citecolor=citecolor,
      }
    % Slightly bigger margins than the latex defaults
    
    \geometry{verbose,tmargin=1in,bmargin=1in,lmargin=1in,rmargin=1in}
    
    

\begin{document}
    
    \maketitle
    
    

    
    \section{Homework 3}\label{homework-3}

    \section{Finite difference methods.}\label{finite-difference-methods.}

    \subsection{Explicit Finite
Difference}\label{explicit-finite-difference}

    Our goal is to value European options with \(V(S, t)\).

    \subsubsection{Solving the PDE}\label{solving-the-pde}

    We will follow the Black-Scholes assumption that the underlying stock
follows this stochastic process:

\[dS_t = rS_t dt + \sigma S_t dW_t\]

Then the price of the European option must satisfy this PDE:

\[\frac{\partial V}{\partial t} + rS \frac{\partial V}{\partial S} + \frac{1}{2} \sigma^2 S^2 \frac{\partial^2 V}{\partial S^2} - rV = 0\]

    In order to solve this PDE, we must have constant coefficients. We can
do this through a change of variable, modeling returns instead of the
actual stock price.

\[S = e^x\] \[x = \ln S\]

Then we will get a new value equation \(u\) where
\[V(S, t) = V(e^x, t) = u(x, t)\]
\[\frac{\partial V}{\partial t}(t, S) = \frac{\partial u}{\partial t}(t, x)\]

And using Ito's lemma, our original Black-Scholes PDE becomes

\[\frac{\partial u}{\partial t} + \nu\frac{\partial u}{\partial x} + \frac{1}{2}\sigma^2 \frac{\partial^2 u}{\partial x^2} - ru = 0\]

where

\[\nu = r - \frac{\sigma^2}{2}\]

    Merton (1973) showed that this PDE, like the heat equation, can be
solved analytically and used to value options through the famous
Black-Scholes equation. But we can also solve it via a numerical method,
the \textbf{explicit finite difference} method.

    \subsubsection{Discretizing the Domain}\label{discretizing-the-domain}

    This process involves discretizing this equation, and solving it
backwards from the payoff at maturity \(T\).

This begins by discretizing our domain.

Our domain is \(t \in [0, T]\) and \(x \in (-\infty, \infty)\).

We will discretize \(t\) into \(n + 1\) points like so:
\[\Delta t = \frac{T}{n}\]
\[t = \{0, \Delta t, 2 \Delta t, \ldots, n \Delta t\}\]

For \(x\), we must set some large boundary instead of using \(\infty\),
which we will define as \(N\). Therefore we will have \(2N + 1\) points
like so:
\[x = \{-N \Delta x, (-N + 1) \Delta x, \ldots, 0, \Delta x, \ldots, N \Delta x\}\]

The value of \(\Delta x\) is technically arbitrary. However, in order
for this process to converge, \(\Delta x\) must follow

\[\Delta x \geq \sigma \sqrt{3 \Delta t}\]

The time complexity of the explicit algorithm is
\(O(\Delta x^2 + \Delta t)\). Since we want to minimize the time
complexity, the best choice of \(\Delta x\) is in practice always
\(\sigma \sqrt{3\Delta t}\).

    \subsubsection{Discretizing the
Derivatives}\label{discretizing-the-derivatives}

    For the explicit finite difference method, there are four points we will
need. These are - \(u_{i+1, j+1}\) - \(u_{i+1, j}\) - \(u_{i+1, j-1}\) -
\(u_{i, j}\)

And there are three derivatives we are trying to calculate
\[\frac{\partial u}{\partial t}, \frac{\partial u}{\partial x}, \frac{\partial^2 u}{\partial x^2}\]

We can use the limit equation for derivatives to describe finite
difference for the first-order derivatives.

\[u'(x) = \lim_{h \rightarrow 0}  \frac{u(x + h) - u(x)}{h}\]

And we can also use Taylor expansion to get the limit equation for
second-order derivatives in terms of the first-order equation.

\[u''(x) = \lim_{h \rightarrow 0} \frac{u(x+h) - 2u(x) + u(x-h)}{h^2}\]

    The derivative with respect to \(t\) most neatly fits into this
paradigm. If we define \[h = \Delta t\] then we get
\[\frac{\partial u}{\partial t} = \frac{u_{i+1, j} - u_{i, j}}{\Delta t}\]

    For the first-order derivative with respect to \(x\), because we are
calculating these values with respect to \(u_{i, j}\), we don't want to
bias it up or down, so we will use the above and below point and then
average them. \[h = \Delta x\]
\[\frac{\partial u}{\partial x} = \frac{u_{i+1,j+1} - u_{i+1,j-1}}{2\Delta x}\]

    And for the second-order derivative with respect to \(x\) we will use
the corresponding limit equation.

\[h = \Delta x\]
\[\frac{\partial^2 u}{\partial x^2} = \frac{u_{i+1,j+1} - 2u_{i+1,j} + u_{i+1,j-1}}{\Delta x^2}\]

    \subsubsection{The Discretized Equation}\label{the-discretized-equation}

    \[\frac{\partial u}{\partial t} + \nu\frac{\partial u}{\partial x} + \frac{1}{2}\sigma^2 \frac{\partial^2 u}{\partial x^2} - ru = 0\]

    Substituting back these finite differences into our original equation,
we get

\[\frac{u_{i+1, j} - u_{i, j}}{\Delta t} + \nu \frac{u_{i+1,j+1} - u_{i+1,j-1}}{2\Delta x} + \frac{1}{2} \sigma^2 \frac{u_{i+1,j+1} - 2u_{i+1,j} + u_{i+1,j-1}}{\Delta x^2} - r u_{i+1, j} = 0\]

    Expand out the equation.
\[\frac{u_{i+1, j}}{\Delta t} - \frac{u_{i, j}}{\Delta t} + \frac{\nu}{2\Delta x} u_{i+1, j+1} - \frac{\nu}{2\Delta x} u_{i+1, j-1} + \frac{\sigma^2}{2\Delta x^2} u_{i+1, j+1} - \frac{\sigma^2}{\Delta x^2} u_{i+1, j} + \frac{\sigma^2}{2 \Delta x^2} h - ru_{i+1, j} = 0\]

Rearrange to solve for \(u_{i, j}\).
\[u_{i, j} = \Delta t \left( \frac{\sigma^2}{2\Delta x^2} u_{i+1, j+1} + \frac{\nu}{2\Delta x} u_{i+1, j+1} \right) + \Delta t \left( \frac{\sigma^2}{2\Delta x^2} u_{i+1, j-1} - \frac{\nu}{2\Delta x} u_{i+1, j-1} \right) + u_{i+1, j} - \Delta t \frac{\sigma^2}{\Delta x^2} u_{i+1, j} - ru_{i+1, j} \Delta t\]

Factor out the probabilities.
\[u_{i, j} = p_u u_{i+1, j+1} + p_m u_{i+1, j} + p_d u_{i+1, j-1}\]

\[p_u = \Delta t \left( \frac{\sigma^2}{2\Delta x^2} + \frac{\nu}{2\Delta x} \right)\]
\[p_m = 1 - \Delta t \frac{\sigma^2}{\Delta x^2} - r \Delta t\]
\[p_d = \Delta t \left( \frac{\sigma^2}{2\Delta x^2} - \frac{\nu}{2\Delta x} \right)\]

    \subsubsection{Valuation}\label{valuation}

    To carry out the valuation, we will follow these steps.

\begin{enumerate}
\def\labelenumi{\arabic{enumi}.}
\tightlist
\item
  Initialize the following constants:

  \begin{itemize}
  \tightlist
  \item
    \(K\)
  \item
    \(T\)
  \item
    \(S\)
  \item
    \(\sigma\)
  \item
    \(r\)
  \item
    \(\delta\)
  \item
    \(n\)
  \item
    \(N\)
  \item
    \(dt\)
  \item
    \(dx\)
  \item
    \(\nu\)
  \item
    \(p_u\)
  \item
    \(p_m\)
  \item
    \(p_d\)
  \end{itemize}
\item
  Create a vector of asset prices at maturity.
\item
  Initialize option values at maturity based on the option payoff
  formula.
\item
  Step backwards through the lattice by solving the discretized equation
  for each point in each time step based on the three points in the next
  time step.
\item
  For the boundary conditions, initialize them based on the option type.
\item
  Return the value at (0, 0)
\end{enumerate}

    \subsection{Implicit Finite
Difference}\label{implicit-finite-difference}

    Use the same discretized equation for explicit finite difference, but
calculate the derivative with respect to \(x\) at time step \(i\) not
\(i + 1\).

\[\frac{u_{i+1, j} - u_{i, j}}{\Delta t} + \nu \frac{u_{i,j+1} - u_{i,j-1}}{2\Delta x} + \frac{1}{2} \sigma^2 \frac{u_{i,j+1} - 2u_{i,j} + u_{i,j-1}}{\Delta x^2} - r u_{i, j} = 0\]

    We can perform the same steps of expansion and factoring.

    Expand out the equation.
\[\frac{u_{i+1, j}}{\Delta t} - \frac{u_{i, j}}{\Delta t} + \frac{\nu}{2\Delta x} u_{i, j+1} - \frac{\nu}{2\Delta x} u_{i, j-1} + \frac{\sigma^2}{2\Delta x^2} u_{i, j+1} - \frac{\sigma^2}{\Delta x^2} u_{i, j} + \frac{\sigma^2}{2 \Delta x^2} h - ru_{i, j} = 0\]

Rearrange to solve for \(u_{i, j}\).
\[u_{i + 1, j} = \Delta t \left( -\frac{\sigma^2}{2\Delta x^2} u_{i, j+1} - \frac{\nu}{2\Delta x} u_{i, j+1} \right) + \Delta t \left( -\frac{\sigma^2}{2\Delta x^2} u_{i, j-1} + \frac{\nu}{2\Delta x} u_{i, j-1} \right) + u_{i, j} + \Delta t \frac{\sigma^2}{\Delta x^2} u_{i, j} - ru_{i, j} \Delta t\]

Factor out the constants (not probabilities!).
\[u_{i + 1, j} = A u_{i, j+1} + B u_{i, j} + C u_{i, j-1}\]

\[A = -\frac{1}{2} \Delta t \left( \frac{\sigma^2}{\Delta x^2} + \frac{\nu}{\Delta x} \right)\]
\[B = 1 + \Delta t \frac{\sigma^2}{\Delta x^2} + r \Delta t\]
\[C = -\frac{1}{2} \Delta t \left( \frac{\sigma^2}{\Delta x^2} - \frac{\nu}{\Delta x} \right)\]

    We can construct a system of equations that represents the time step
from \(t_i\) to \(t_{i + 1}\). For each column \(V\) which has
\(2N + 1\) rows,

\[A V_{i, N - 1} + B V_{i, N - 1} + C V_{i, N - 1} = V_{i, N - 1}\]
\[A V_{i, N - 2} + B V_{i, N - 2} + C V_{i, N - 2} = V_{i, N - 2}\]
\ldots{}
\[A V_{i, -N + 1} + B V_{i, -N + 1} + C V_{i, -N + 1} = V_{i, -N + 1}\]

Note here that because the value of each point depends on the past above
and below points, these equations do not work for the boundaries at the
top and bottom.

    For these, we set the boundary condition based on the delta of the
option as the stock price goes to infinity/zero. The finite difference
of the delta here is calculated at the top as
\[V_{i, N} - V_{i, N - 1} = \lambda_U\] and at the bottom as
\[V_{i, -N + 1} - V_{i, -N} = \lambda_L\]

The value of \(\lambda_U\)/\(\lambda_L\) depends on the option type.

For a call option, as the stock price approaches infinity, the payoff
from the option just keeps adding on the stock price, giving them a
linear relationship with a delta of 1.
\[\lambda_{Uc} = S \uparrow \infty: \frac{\partial V_c}{\partial S} = 1\]
And as the stock price approaches zero, the call will never be exercised
and the payoff doesn't change, giving a delta of 0.
\[\lambda_{Lc} = S \downarrow 0: \frac{\partial V_c}{\partial S} = 0\]

For a put option, as the stock price approaches infinity, the put will
never be exercised giving a delta of 0.
\[\lambda_{Up} = S \uparrow \infty: \frac{\partial V_p}{\partial S} = 0\]

And as the stock price approaches 0, the put payoff will have the same
linear relationship with the call, but it increases when the price
decreases, making the delta -1.

\[\lambda_{Dp} = S \downarrow 0: \frac{\partial V_p}{\partial S} = -1\]

    Putting all of this together, we can construct a matrix equation of form
\(Ax = b\) to represent this systems of equations.

\[\begin{bmatrix} 1 & - 1 & 0 & 0 & 0 & \ldots & 0 \\ A & B & C & 0 & 0 & \ldots & 0 \\ 0 & A & B & C & 0 & \ldots & 0 \\ \vdots & \ddots & \ddots & \ddots & \ddots & \ddots & \vdots \\ 0 & 0 & \ddots & \ddots & B & C & 0 \\ 0 & 0 & 0 & \ddots & A & B & C \\ 0 & 0 & 0 & \ldots & 0 & 1 & -1 \end{bmatrix} \begin{bmatrix} V_{i, N} \\ V_{i, N - 1} \\ V_{i, N - 2} \\ \vdots \\ \vdots \\ V_{i, -N + 1} \\ V_{i, -N} \end{bmatrix} = \begin{bmatrix} \lambda_U \\ V_{i+1,N-1} \\ V_{i+1, N-2} \\ \vdots \\ \vdots \\ V_{i+1,-N+1} \\ \lambda_L \end{bmatrix}\]

Normally, this calculation would be done by inverting the matrix.
However, for sufficiently large \(N\), this becomes computationally
infeasible. Fortunately, there is a recurrence procedure for calculating
tridiagonal matrices of these forms in \(O(n)\).

    The procedure is derived in the following way.

For simplicity of notation, I will describe \(V_{i, j}\) by \(x_j\) and
\(V_{i, + 1, j}\) as \(y_j\) and index from 1 to \(n\).

The first equation is \[x_1 - x_2 = y_1\] which we can rearrange to
solve for \(x_1\) in terms of \(x_2\) \[x_1 = y_1 + x_2\] I can perform
a change of variable here where I define \[D_1 = y_1\] \[E_1 = 1.0\] so
that \[x_1 = D_1 + E_1 x_2\] This is the initial condition for the
recurrence, as we will see.

The second equation is \[Ax_1 + Bx_2 + Cx_3 = y_2\] We can substitute in
the first equation like so \[A(y_1 + x_2) + Bx_2 + Cx_3 = y_2\] and
rearrange to solve for \(x_2\) in terms of \(x_3\)
\[Ay_1 + Ax_2 + Bx_2 + Cx_3 = y_2\]
\[x_2 = \frac{y_2 - Ay_1}{A + B} - \frac{C}{A + B}x_3\] Doing the same
of change of variable, we can define
\[D_2 = \frac{y_2 - AD_1}{AE_1 + B}\] \[E_2 = -\frac{C}{AE_1 + B}\] so
that \[x_2 = D_2 + E_2 x_3\] We can see the same structure emerge here.
One more example will fully illuminate it.

The third equation is \[Ax_2 + Bx_3 + Cx_4 = y_3\] Substituting in the
second equation, \[A(D_2 + E_2 x_3) + Bx_3 + Cx_4 = y_3\] Then
rearranging to solve for \(x_3\) in terms of \(x_4\)
\[AD_2 + AE_2 x_3 + Bx_3 + Cx_4 = y_3\]
\[x_3 = \frac{y_3 - AD_2}{AE_2 + B} - \frac{C}{AE_2 + B}x_4\] With the
change of variable, \[D_3 = \frac{y_3 - AD_2}{AE_2 + B}\]
\[E_3 = -\frac{C}{AE_2 + B}\]

The recurrence now becomes clear. Each equation follows this form.
\[x_j = D_j + E_j x_{j+1}\] where
\[D_j = \frac{y_j - AD_{j - 1}}{AE_{j - 1} + B}\]
\[E_j = -\frac{C}{AE_{j - 1} + B}\]

We have our initial condition, our recurrence, but where's our stopping
point? The second-to-last equation, expressed in terms of \(D\) and
\(E\), is \[x_{n-1} = D_{n-1} + E_{n-1} x_n\]

And the last equation is \[x_{n-1} - x_n = y_n\] Substituting the
previous equation into this one, we get
\[D_{n-1} + E_{n-1} x_n - x_n = y_n\] Then rearranging to solve for
\(x_n\) \[x_n = \frac{y_n - D_{n-1}}{E_{n-1} - 1.0}\] which is a
complete solution.

    The tridiagonal system solver procedure then follows like so:

\begin{enumerate}
\def\labelenumi{\arabic{enumi}.}
\tightlist
\item
  Create vectors for \(D\) and \(E\) of size \(n\) and set their first
  element to \(y_1\) and \(1.0\), respectively.
\item
  Forward-substitute the values for \(D\) and \(E\) using the \(D\) and
  \(E\) recurrences.
\item
  Create a vector for \(x\) and solve for \(x_n\) directly.
\item
  Back-substitute each element in \(x\) through the \(x\) recurrence.
\end{enumerate}

    This solver is run at each time step in the same way that explicit
finite difference is, and then the final option price is the value of
\(V_{0, 0}\).

    \section{Choosing Parameter Values}\label{choosing-parameter-values}

    \subsection{Explicit Finite
Difference}\label{explicit-finite-difference}

    For explicit finite difference, as discussed previously, the best value
for \(\Delta x\) is always

\[\Delta x = \sigma \sqrt{3 \Delta t}\]

Our order of convergence is \(O(\Delta x^2 + \Delta t)\), so our goal of
converging to \(\epsilon\) is satisfied by
\[\Delta x^2 + \Delta t = \epsilon\] which we can substitute and
rearrange to find the correct value for \(\Delta t\)
\[\sigma^2 3 \Delta t + \Delta t = \epsilon\]
\[\Delta t = \frac{\epsilon}{1 + 3 \sigma^2}\]

The value for \(n\) naturally falls out of this via
\(n = \frac{T}{\Delta t}\) as for all these procedures. And we can
define \(N = n\) (although I can't find a reason to do this beyond ``the
nature of the explicit scheme'', either in Mariani \& Florescu (2019) or
Clewlow (1998)).

    \subsection{Implicit Finite
Difference}\label{implicit-finite-difference}

    The order of convergence is the same as with explicit finite difference,
but the method is unconditionally stable and convergent. We can
arbitrarily choose for the error to be divided equally between
\(\Delta t\) and \(\Delta x\).

\[\Delta t = \frac{\epsilon}{2}\]
\[\Delta x = \sqrt{\frac{\epsilon}{2}}\]

\(n\) is computed as before, and we choose \(N = n\) to be the same
order of magnitude as \(n\) as suggested by Mariani \& Florescu (2019).
(M \& F cites Clewlow (1998) for this but I can't find any source for
that in there.)

    \section{Calculation}\label{calculation}

    \begin{itemize}
\tightlist
\item
  Explicit Method - European Call Price: 9.72838461521403

  \begin{itemize}
  \tightlist
  \item
    \(n\): 11200
  \item
    \(N\): 11200
  \end{itemize}
\item
  Explicit Method - European Put Price: 5.884952006728757

  \begin{itemize}
  \tightlist
  \item
    \(n\): 11200
  \item
    \(N\): 11200
  \end{itemize}
\item
  Implicit Method - European Call Price: 9.72734157812497

  \begin{itemize}
  \tightlist
  \item
    \(n\): 20000
  \item
    \(N\): 20000
  \end{itemize}
\item
  Implicit Method - European Put Price: 5.883910596683879

  \begin{itemize}
  \tightlist
  \item
    \(n\): 20000
  \item
    \(N\): 20000
  \end{itemize}
\end{itemize}

    \section{Empirical Convergence}\label{empirical-convergence}

    There was some error in my implicit finite difference code that caused
it to fail to converge.


    % Add a bibliography block to the postdoc
    
    
    
\end{document}
